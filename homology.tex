%%%%%%%%%%%%%%%%%%%%%%%%%%%%%%%%%%%%%%%%%
% kaobook
% LaTeX Class
% Version 0.9.8 (2021/08/23)
%
% This template originates from:
% https://www.LaTeXTemplates.com
%
% For the latest template development version and to make contributions:
% https://github.com/fmarotta/kaobook
%
% Authors:
% Federico Marotta (federicomarotta@mail.com)
% Based on the doctoral thesis of Ken Arroyo Ohori (https://3d.bk.tudelft.nl/ken/en)
% and on the Tufte-LaTeX class.
% Modified for LaTeX Templates by Vel (vel@latextemplates.com)
%
% License:
% LPPL (see included MANIFEST.md file)
%
%%%%%%%%%%%%%%%%%%%%%%%%%%%%%%%%%%%%%%%%%

%----------------------------------------------------------------------------------------
%	EXAMPLE AND DOCUMENTATION OF THE KAOBOOK CLASS
%----------------------------------------------------------------------------------------

\documentclass[
    a4paper,
    fontsize=10pt,
    twoside=true, 			% Use different layouts for even and odd pages (in particular, if twoside=true, the margin column will be always on the outside)
	open=any, 				% If twoside=true, uncomment this to force new chapters to start on any page, not only on right (odd) pages
	%chapterentrydots=true, % Uncomment to output dots from the chapter name to the page number in the table of contents
	numbers=noenddot,	 	% Comment to output dots after chapter numbers; the most common values for this option are: enddot, noenddot and auto (see the KOMAScript documentation for an in-depth explanation)
]{kaobook}

%----------------------------------------------------------------------------------------
%	PACKAGES AND OTHER DOCUMENT CONFIGURATIONS
%----------------------------------------------------------------------------------------

% Choose the language
\ifxetexorluatex
	\usepackage{polyglossia}
	\setmainlanguage{spanish}
\else
	\usepackage[spanish]{babel} % Load characters and hyphenation
\fi
\usepackage[spanish=spanish]{csquotes}	% English quotes

% Load packages for testing
%\usepackage{blindtext} 	% Print text without any meaning for testing purposes
%\usepackage{showframe} 	% Uncomment to show boxes around the text area, margin, header and footer
%\usepackage{showlabels} 	% Uncomment to output the content of \label commands to the document where they are used

% Load the bibliography package
\usepackage[backend=bibtex]{kaobiblio}
\addbibresource{chapters/references} % Bibliography file

% Load mathematical packages for theorems and related environments
\usepackage[framed=true]{kaotheorems}

% Load the package for hyperreferences
\usepackage{kaorefs}

% Paths in which to look for images
\graphicspath{{figures/}}

% Make LaTeX produce the files required to compile the index
\makeindex[columns=3, title=Alphabetical Index, intoc]

%\makeglossaries % Make LaTeX produce the files required to compile the glossary
%\input{chapters/glossary.tex} % Include the glossary definitions

\makenomenclature % Make LaTeX produce the files required to compile the nomenclature

% Reset sidenote counter at chapters
%\counterwithin*{sidenote}{chapter}

% Split figure into subfigures
\usepackage{subcaption}

\usepackage{amsmath,amssymb}
\usepackage{mathrsfs}

\usepackage{subcaption}

% Index of terms
\usepackage{imakeidx}
\newcommand{\define}[1]{\textbf{#1}\index{#1}}

%Hyperlinks
\usepackage{hyperref}

\usepackage{algebra}
\usepackage{singtheo}
\usepackage{nicetikz}
\usepackage{fonts}
\usepackage{tables}

%----------------------------------------------------------------------------------------

\begin{document}

%\input{title}
\frontmatter % Denotes the start of the pre-document content, uses roman numerals

\makeatletter
%\input{frontmatter/opening}
\makeatother

\makeatletter
%\makeatletter
\uppertitleback{\@titlehead} % Header

\lowertitleback{
	\textbf{Disclaimer}\\
	La plantilla utilizada ha sido creada por Federicco Marotta y está basada en la tesis doctoral de Ken Arroyo Ohmori.
	
	\medskip
	
	\textbf{Membrete} \\
	Este documento ha sido creado con la ayuda de \href{https://sourceforge.net/projects/koma-script/}{\KOMAScript} y \href{https://www.latex-project.org/}{\LaTeX} mediante la clase \href{https://github.com/fmarotta/kaobook/}{kaobook}.
	
	La plantilla utilizada se puede encontrar en: \\\url{https://github.com/fmarotta/kaobook}
	
	(¡El autor te invita a contribuir!)
%	
%	\medskip
%	
%	\textbf{Publisher} \\
%	First printed in May 2019 by \@publishers
}
\makeatother
\makeatother

%\input{frontmatter/dedication}

%\maketitle

%----------------------------------------------------------------------------------------
%	PREFACE
%----------------------------------------------------------------------------------------

%\input{chapters/preface.tex}
\index{preface}

%----------------------------------------------------------------------------------------
%	TABLE OF CONTENTS & LIST OF FIGURES/TABLES
%----------------------------------------------------------------------------------------

\begingroup % Local scope for the following commands

% Define the style for the TOC, LOF, and LOT
%\setstretch{1} % Uncomment to modify line spacing in the ToC
%\hypersetup{linkcolor=blue} % Uncomment to set the colour of links in the ToC
\setlength{\textheight}{230\hscale} % Manually adjust the height of the ToC pages

% Turn on compatibility mode for the etoc package
\etocstandarddisplaystyle % "toc display" as if etoc was not loaded
\etocstandardlines % "toc lines as if etoc was not loaded

\tableofcontents % Output the table of contents

\listoffigures % Output the list of figures

% Comment both of the following lines to have the LOF and the LOT on 
% different pages
\let\cleardoublepage\bigskip
\let\clearpage\bigskip

\listoftables % Output the list of tables

\listoflstlistings % Output the list of listings

\endgroup

%----------------------------------------------------------------------------------------
%	MAIN BODY
%----------------------------------------------------------------------------------------

\mainmatter % Denotes the start of the main document content, resets page numbering and uses arabic numbers
\setchapterstyle{kao} % Choose the default chapter heading style

\chapter*{Introducción}
En el año 1758, Euler publica un artículo que cubre diversas propiedades de
los poliedros. El principal resultado de su artículo es la celebrada fórmula
de Euler para poliedros convexos,
\[V-A+C=2\]
La demostración de este resultado se basa en el hecho de que los poliedros
convexos son homeomorfos a un sólido común, la bola cerrada. Si consideramos
un poliedro irregular que sea homeomorfo a la esfera, este resultado sigue
siendo válido.

Sin embargo, podemos encontrar poliedros que no verifiquen esta fórmula
eliminando la condición de convexidad. Un ejemplo de poliedro que no verifica
esta expresión es el tetrahemihexaedro, con 6 vértices, 12 aristas y 7 caras.
Si computamos su valor $V-A+C$, obtenemos
\[6-12+7=1\]
El valor $V-A+C$ de un poliedro regular se denomina \emph{característica de
Euler} del poliedro.

\begin{marginfigure}
\includegraphics{Figures/Tetrahemihexahedron.png}
%https://en.wikipedia.org/wiki/Euler_characteristic#/media/File:Tetrahemihexahedron.png
\caption[Tetrahemihexaedro]{Tetrahemihexaedro regular. Algunas de sus caras se
intersecan entre sí, haciendo que su topología sea diferente a la de la bola
cerrada. Imagen: \cite{Tetra}.}
\end{marginfigure}

Un espacio topológico $X \subset \mb{R}^n$ 2AN y Hausdorff es una superficie
si, dado $p \in X$, existe un entorno abierto $U \subset X$ de $p$ homeomorfo
a una bola de $\mb{R}^2$. Desde un punto de vista geométrico, podemos decir
que los puntos de $X$ perciben el mundo en dos dimensiones, al igual que los
personajes de la novela \emph{Planilandia}. Algunas superficies pueden ser
representadas utilizando poliedros regulares, que podemos clasificar en
función de su característica de Euler. Cuando una superficie sea homeomorfa a
algún poliedro regular, diremos que es poliédrica.

A partir de un espacio topológico, podemos generar una familia de grupos
abelianos llamados \emph{grupos de homología}. Los grupos de homología nos
permiten utilizar técnicas de álgebra conmutativa para conocer algunas de las
propiedades topológicas de un espacio, permitiendo probar resultados que están
fuera del alcance de la topología conjuntista. En particular, veremos una
demostración del teorema del punto fijo de Brouwer, que establece la
existencia de puntos fijos para cualquier aplicación continua entre conjuntos
convexos.

Este texto está fuertemente basado en \cite{Vick94}, un texto dirigido a
estudiantes de máster y doctorado (conocido en Estados Unidos como el
\emph{graduate level}), por lo que las explicaciones son más breves y muchos
detalles se asumen triviales. Mi objetivo es adaptar estos textos, de
forma que sean lo más asequible posible a estudiantes de 3º y 4º de carrera.

Dado que la información en \cite{Vick94} está muy concentrada, se han tomado
los dos primeros capítulos y convertido en partes. Se recomienda al lector
tratar de entender y computar ejemplos antes de pasar a la parte siguiente.

La primera parte de este texto corresponde al capítulo 1, \emph{Singular
Homology Theory}, donde se introducen los grupos de homología singular y las
sucesiones de Mayer-Vietoris. Las sucesiones de Mayer-Vietoris son la técnica
básica para calcular los grupos de homología de un espacio topológico, y
son válidas para cualquier espacio.

La segunda parte corresponde al capítulo 2, \emph{Attaching Spaces with Maps},
donde se introducen los espacios CW-complejos y los grupos de homología
celular. La homología celular es una forma más directa de computar los grupos
de homología, pero requiere que nuestro espacio admita una estructura especial,
la \emph{estratificación por CW-complejos}.

Finalmente, los apéndices presentan información adicional que no es necesaria
para poder seguir el texto, pero he considerado interesante y digma de
discusión.

El objetivo de este texto es introducir al lector en la teoría de homología
singular, y está dirigido principalmente a estudiantes de la \emph{Universitat
de València}. Como consecuencia, el lector se asume familiarizado con el
contenido cubierto por las asignaturas de Estructuras algebraicas y Topología
de segundo curso.


\pagelayout{wide} % No margins
\addpart{Homología singular de un espacio topológico}
\pagelayout{margin} % Restore margins
\setchapterpreamble[u]{\margintoc}

\chapter{Grupos de homología singular}
\section{Símplices singulares}
Decimos que una familia de puntos $S=\{x_0, \dots, x_p\} \subset \mb{R}^n$ es \textbf{afínmente independiente} si, dados $\lambda_1,\dots,\lambda_n, \mu_1,\dots,\mu_n \in \mb{R}$ con
	\begin{align*}
		\sum^p_{i=0}\lambda_ix_i=\sum^p_{i=0}\mu_ix_i; && \sum^p_{i=0}\lambda_i=\sum^p_{i=0}\mu_i
	\end {align*}
se verifica que $\lambda_i=\mu_i$ para todo $1\leq i \leq n$.

\begin{proposition}
	La familia de puntos $S=\{x_0, \dots, x_p\} \subset \mb{R}^n$ es afínmente independiente si y sólo si la familia de vectores $\{x_1-x_0,\dots,x_p-x_0\}$ es linealmente independiente.
\end{proposition}

\begin{proof}
	Sea $S$ afínmente independiente y $\lambda_1,\dots, \lambda_p$ tales que
	\begin{equation}\label{CombLinNula}
		\sum^p_{i=1}\lambda_i(x_i-x_0)=0 \iff \sum^p_{i=1}\lambda_ix_i=x_0\sum^p_{i=1}\lambda_i
	\end{equation}
	Definimos $\lambda_0=-\lambda_1-\dots-\lambda_p$.
	Por construcción, $\lambda_0+\lambda_1+\dots+\lambda_p=\lambda_0-\lambda_0=0$.
	Aplicando la identidad \eqref{CombLinNula},
		\[0=\sum^p_{i=1}\lambda_i(x_i-x_0)=\sum^p_{i=1}\lambda_i x_i-x_0\sum^p_{i=1}\lambda_i=\sum^p_{i=0}\lambda_i x_i\]
	Dado que $S$ es una familia afínmente independiente, $\lambda_i=0$ para todo $i=1,\dots,n$.
	Por tanto, $T$ es linealmente independiente.
	Recíprocamente, sea $T$ linealmente independiente y $\mu_0,\alpha_0,\dots,\mu_p,\alpha_p \in \mb{R}$ tales que
	\begin{align*}
		\sum_{i=0}^p \mu_ix_i=\sum_{i=0}^p \alpha_ix_i; && \sum_{i=0}^p \mu_i=\sum_{i=0}^p \alpha_i
	\end{align*}
	Combinando ambas igualdades, deducimos que
	\begin{align*}
	0=\sum^p_{i=0}(\mu_i-\alpha_i)x_i
		&=\sum^p_{i=0}(\mu_i-\alpha_i)x_i-0x_0=\\
		&=\sum^p_{i=0}(\mu_i-\alpha_i)x_i-
		\left(\sum^p_{i=0}(\mu_i-\alpha_i)\right)x_0=\\
		&=\sum^p_{i=1}(\mu_i-\alpha_i)(x_i-x_0)
	\end{align*}
	Dado que $T$ es linealmente independiente, se sigue que
		\[\mu_i-\alpha_i=0 \iff \mu_i=\alpha_i\]
	para $i=1,\dots,p$.
	Por tanto, $S$ es afínmente independiente.
\end{proof}

\begin{marginfigure}
\begin{tikzpicture}[scale=0.6]
	%Puntos
	\draw (3,3)[fill=black] circle (1pt);
	\draw (3,3.5) node {$a$};

	\draw (0,0)[fill=black] circle (1pt);
	\draw (0,-.5) node {$b$};

	\draw (-3,3)[fill=black] circle (1pt);
	\draw (-3,3.5) node {$c$};

	%Vectores
	\draw[-to] (0,0) -- (1,1);
	\draw[-to] (0,0) -- (-1,1);

	%Líneas
	\draw[dashed] (-1,-1) -- (4,4);
	\draw (2,1.5) node {$L_1$};

	\draw[dashed] (1,-1) -- (-4,4);
	\draw (-2,1.5) node {$L_2$};
\end{tikzpicture}
\labfig{puntos independendientes}
\caption[Puntos afínmente independientes]{La independencia afín es el equivalente en conjuntos afines a la independencia lineal.}
\end{marginfigure}

Sea $S \subset \mb{R}^n$ afínmente independiente y $(a,b,c)$ una terna en $S$ con $a\neq b \neq c$.
Si $S$ es afínmente independiente, $a-b$ y $b-c$ son linealmente independientes.
Como $a-b\neq 0$, existen rectas afines $L_1,L_2 \subset \mb{R}^n$ que son paralelas a $a-b$ y $b-c$.
Por independencia lineal, $L_1$ y $L_2$ se cruzan a lo sumo en un punto, por lo que $a \not\in L_2$ y $c \not\in L_1$ (ver \reffig{puntos independientes}).

\begin{definition}
	Un \textbf{$p$-símplice} es la envoltura convexa de una familia afínmente independiente $\{x_0, \dots,x_p\} \subset \mb{R}^n$.
	Dichos puntos reciben el nombre de \textbf{vértices} del símplice.
\end{definition}

\marginnote{
\begin{kaobox}[frametitle=Envoluta convexa]
	Un $C \subseteq \mb{R}^n$ no vacío es convexo si, dados $x,y \in C$, $C$ contiene al segmento $[x,y]=\{xt+y(1-t)\colon t \in [0,1]\}$.
	Si $A \subset \mb{R}^n$ es no vacío, se define su \textbf{envoltura convexa} como el menor subconjunto convexo que contiene a $A$,
		\[\con(A)=\bigcup_{x,y \in A}[x,y]\]
\end{kaobox}
}

\begin{proposition}\labprop{BaseVertices}
	Sea $S\subset \mb{R}^n$ un $p$-símplice y $\{x_0,\dots x_p\}$ una familia afínmente independiente de puntos contenidos en $S$.
	Dado $x \in S$, existen únicos $t_0,\dots, t_p \in [0,1]$ tales que
	\begin{align*}
		x=\sum^p_{i=0}t_ix_i; && \sum^p_{i=0}t_i=1
	\end{align*}
\end{proposition}

Decimos que un $p$-símplice está \textbf{ordenado} cuando se aplica una relación de orden determinada sobre sus vértices.

\begin{example}
	Sean $a=(1,0,0)$, $b=(0,1,0)$ y $c=(0,0,1)$.
	La envoltura convexa de $\{a,b,c\}$ es un 2-símplice $\sigma$.
	Si $e_1=a$, $e_2=b$ y $e_3=c$, la relación de orden $e_i \leq e_j \leftrightharpoons i \leq j$ hace que $\sigma$ sea un 2-símplice ordenado.
\end{example}

\begin{marginfigure}
\begin{tikzpicture}[scale=1.4]
	% X: Left	/ Right
	% Y: Up		/ Down
	% Z: Front	/ Back

	% Tetrahedron
	\draw[fill=blue!30, draw opacity=0] (1,0,0) -- (0,1,0) -- (0,0,0) -- cycle;
	\draw[fill=blue!30, draw opacity=0] (1,0,0) -- (0,0,1) -- (0,0,0) -- cycle;
	\draw[fill=blue!30, draw opacity=0] (0,1,0) -- (0,0,1) -- (0,0,0) -- cycle;
	\draw[dashed] (0,0,0) -- (0,0,1);
	\draw[dashed] (0,0,0) -- (0,1,0);
	\draw[dashed] (0,0,0) -- (1,0,0);
	\draw (0,1,0) -- (0,0,1) -- (1,0,0) -- cycle;

	% Triangle
	\draw[fill=blue!30] (3,0,0) -- (2,0,1) -- (2,1,0) -- cycle;
\end{tikzpicture}
\caption[Triángulo y tetraedro]{Los triángulos y tetraedros constituyen ejemplos de símplices.
Se puede definir una teoría de homología equivalente a la homología singular utilizando sólo símplices, llamada \emph{homología simplicial} (véase \cite{Hatcher}).}
\labfig{123Simplice}
\end{marginfigure}

Sea $\{e_1, \dots, e_{n+1}\}$ la base canónica de $\mb{R}^{n+1}$.
La envoltura convexa de los puntos $\{e_1, \dots, e_{n+1}\}$ es un $n$-símplice, que denotaremos como $\sigma_n$.
Por \refprop{BaseVertices}, los puntos de $\sigma_n$ son de la forma
\begin{align*}
	(t_1,\dots,t_{n+1}) \in \mb{R}^{n+1}; && t_1+\dots+t_{n+1}=1
\end{align*}

Sea $S=\{x_0,\dots,x_n\}\subset \mb{R}^m$ una familia de puntos afínmente independientes: la aplicación continua
\begin{funcion*}
	f\colon \sigma_n \arrow[r]           & \con(S)            \\
	{(t_0,\dots,t_n)} \arrow[r, maps to] & \displaystyle\sum^n_{i=0}t_ix_i
\end{funcion*}
establece una biyección entre $\sigma_n$ y $\con(S)$.
Dado que $\sigma_n$ es compacto y $\con(S)$ es un espacio de Hausdorff, $f$ es un homeomorfismo \sidecite{IntroTopo}, por lo que $\con(S)$ es homeomorfo a $\sigma_n$.
De aquí se sigue que todo $p$-símplice de $\mb{R}^{p+1}$ es homeomorfo a $\sigma_p$, por lo que se conoce como \textbf{$p$-símplice estándar}.

\begin{example}
	\labexample{simplices low dimensions}
	Los símplices en bajas dimensiones son $\sigma_0=\{1\}$, $\sigma_1=[e_1,e_2]$, $\sigma_2$ (el triángulo) y $\sigma_3$ (el tetraedro).
\end{example}

\begin{definition}
	Un \textbf{$p$-símplice singular} en un espacio topológico $X$ es una aplicación continua $\phi\colon \sigma_p \to X$.
\end{definition}

\begin{example}\labexample{S1Complejo}
	Sea $\mb{D}$ la circunferencia unidad en $\mb{C}$.
	La aplicación
	\begin{funcion*}
		\phi\colon \sigma_1 \arrow[r]	& \mb{D}\\
		(t_0,t_1) \arrow[maps to,r] 	& e^{\pi i t_0}
	\end{funcion*}
	es un símplice singular que envía al $1$-símplice estándar en el hemisferio norte de $\mb{D}$ (ver \reffig{Circunferencia}).
	Notemos que $\phi(\sigma_1)$ no es un segmento desde el punto de vista geométrico porque está curvado, pero sí lo es desde un punto de vista topológico, ya que $\phi$ es un homeomorfismo y $\sigma_1$ es un segmento.
\end{example}

\begin{marginfigure}
	\input{figures/S1_Simplex}
	\caption[Circunferencia]{La curva $\phi(\sigma_1)$ dada por el símplice singular $\phi$ de \refexample{S1Complejo}.
	Los símplices singulares amoldan el símplice estándar al espacio de llegada usando su continuidad.}
	\labfig{Circunferencia}
\end{marginfigure}

Dado un $p \in X$ y un espacio topológico $Y$, toda aplicación constante $\cte_p\colon Y \to X$ se identifica con el $0$-símplice singular $\phi_p\colon \sigma_0 \to X$ que envía al punto $1$ en $p$.
En general, todo camino $\alpha\colon [0,1] \to X$ se identifica con el $1$-símplice singular
\begin{funcion*}
\psi\colon	\sigma_1 \arrow[r]             & X      \\
		{(t_0,t_1)} \arrow[r, maps to] & \alpha(t_0)
\end{funcion*}

Sean $X$, $Y$ espacios topológicos.
Dada una aplicación continua $f\colon X \to Y$ y un $p$-símplice singular $\phi\colon \sigma_p \to X$, la aplicación $f_\#(\phi):=f\circ \phi\colon \sigma_p$ es un $p$-símplice singular en $Y$ por ser composición de aplicaciones continuas.
Esto hace que toda aplicación continua dé lugar a una aplicación que envía $p$-símplices singulares de $X$ en $p$-símplices singulares de $Y$.

\begin{proposition}\labprop{ComposicionAlmohadilla}
	\begin{enumerate}
	\item Si $\id$ es la identidad en $X$, $\id_\#$ es la identidad en $\sigma_n$;
	\item si $f\colon X \to Y$ y $g\colon Y \to Z$ son continuas, $(g\circ f)_\#=g_\#\circ f_\#$.
	\end{enumerate}
\end{proposition}

Sea $\phi\colon \sigma_p \to X$ un $p$-símplice singular y $0 \leq i \leq p$.
Se define la cara $i$-ésima de $\phi$ como el $(p-1)$-símplice singular
\begin{funcion*}
	\p_{(i)}\phi\colon \sigma_{p-1} \arrow[r] & X \\
	{(t_0,\dots,t_{p-1})} \arrow[r, maps to] & \phi(t_0,\dots,t_{i-1},0,t_{i},\dots,t_{p-1})
\end{funcion*}

\begin{example}
<<<<<<< HEAD
	El 2-símplice singular
	\begin{funcion*}
		\phi\colon  \sigma_2 \arrow[r] & \mb{R}^3 \\
		(x,y,z) \arrow[r, maps to] & (2x+2,2y+2,2z+2)
	\end{funcion*}
	tiene como caras
	\begin{align*}
		\p_{(0)}\phi(u,v)&=\phi(0,u,v)=(2,2u+2,2v+2)\\
		\p_{(1)}\phi(u,v)&=\phi(u,0,v)=(2u+2,2,2v+2)\\
		\p_{(2)}\phi(u,v)&=\phi(u,v,0)=(2u+2,2v+2,2)
	\end{align*}
	Desde un punto de vista geométrico, $\p_{(i)}\phi(\sigma_1)$ es una cara del tetraedro $\phi(\sigma_2)$.
=======
Considérese el 2-símplice singular
\begin{funcion}
	\phi\colon  \sigma_2 \arrow[r] & \mb{R}^3 \\
	(x,y,z) \arrow[r, maps to] & (2x+2,2y+2,2z+2)
\end{funcion}

Las caras de $\phi$ son los $1$-símplices singulares
\begin{align*}
	\p_{(0)}\phi(u,v)&=\phi(0,u,v)=(2,2u+2,2v+2)\\
	\p_{(1)}\phi(u,v)&=\phi(u,0,v)=(2u+2,2,2v+2)\\
	\p_{(2)}\phi(u,v)&=\phi(u,v,0)=(2u+2,2v+2,2)
\end{align*}

Desde un punto de vista geométrico, $\p_{(i)}\phi(\sigma_1)$ son las caras del tetraedro $\phi(\sigma_2)$ para $i=0,1,2$.
>>>>>>> f4da141 (removing conflicts 2)
\end{example}

\section{Grupos de homología}
Dado un grupo abeliano $G$, consideramos la acción $\rho\colon G\times \mb{Z} \to G$ dada por
\begin{align*}
<<<<<<< HEAD
	ng:=g+\stackrel{(n)}{\dots}+g	&& ng:=-g-\stackrel{(n)}{\dots}-g\\
	(n > 0)							&& (n < 0)
=======
	ng:=\sum^n_{j=0}g	&& ng:=\sum^{-n}_{j=0}-g\\
	(n>0)				&& (n < 0)
>>>>>>> f4da141 (removing conflicts 2)
\end{align*}
Decimos que un subconjunto $S \subset G$ es un \textbf{sistema generador} de $G$ si, dado un $g \in G$, podemos hallar $b_1,\dots,b_n \in S$ y enteros $\mu_1,\dots,\mu_n$ tales que
\[g=\mu_1b_1+\mu_2b_2+\dots+\mu_nb_n\]

\begin{definition}
<<<<<<< HEAD
	Sea $G$ un grupo abeliano.
	Un sistema generador $B$ de $G$ es una \textbf{base} si, dados $b_1,\dots,b_n \in B$ y enteros $\lambda_1,\dots,\lambda_n$,
	\begin{equation}
		\label{SisLibre} \sum^n_{i=1}\lambda_ib_i=0 \implies \lambda_i=0
	\end{equation}
	Decimos que $G$ es un \textbf{grupo libre} si admite una base.
\end{definition}

\begin{example}
	Supongamos que el grupo aditivo $\mb{Q}$ es un grupo libre generado por un cierto conjunto $B \subseteq \mb{Q}$.
	Dados $\frac{a}{b}, \frac{c}{d} \in B$,
		\[cb\frac{a}{b}-ad\frac{c}{d}=0,\]
	por lo que $B$ está formado por un único elemento.
	Sea $p$ un número primo coprimo con $b$.
	Por ser $\mb{Q}$ libre, existirá un entero $\alpha$ tal que
		\[\frac{1}{p}=\alpha\frac{a}{b}\]
	pero esto implica que $b=\alpha pa$, en contradicción con la primalidad de $p$.
	Por tanto, $\mb{Q}$ no es un grupo libre.
\end{example}

\marginnote[-2.2cm]{
\begin{kaobox}[frametitle=Soporte de una aplicación]
	Dado un grupo abeliano $G$, el soporte de una aplicación $f\colon A \to G$ es el conjunto de los puntos donde no se anula.
\end{kaobox}
}

Se define el \textbf{grupo libre} generado por $A$ como la familia $\mc{F}(A)$ de todas las aplicaciones $f\colon A \to \mb{Z}$ con soporte finito, dotado con la suma de aplicaciones.

\begin{proposition}
	El grupo $\mc{F}(A)$ es libre.
=======
Sea $G$ un grupo abeliano. Un sistema generador $B$ de $G$ es una \textbf{base}
si, dados $b_1,\dots,b_n \in B$ y enteros $\lambda_1,\dots,\lambda_n$,
\begin{align}
	\label{SisLibre} \sum^n_{i=1}\lambda_ib_i=0 \implies \lambda_i=0
\end{align}
Decimos que $G$ es un \textbf{grupo libre} si admite una base.
\end{definition}

\begin{example}
	Supongamos que $\mb{Q}^+=(\mb{Q},+)$ es un grupo libre generado por un cierto
	conjunto $B \subseteq \mb{Q}$. Dados $\frac{a}{b}, \frac{c}{d} \in B$, se tiene
	que
		\[cb\frac{a}{b}-ad\frac{c}{d}=0\]
	por lo que $B$ está formado por un único elemento.

	Sea $p$ un número primo coprimo con $b$. Por ser $\mb{Q}^+$ libre, existirá un
	entero $\alpha$ tal que
		\[\frac{1}{p}=\alpha\frac{a}{b}\]
	pero esto implica que $b=\alpha pa$, en contradicción con la premisa de que $p$
	es coprimo con $b$. Por tanto, $\mb{Q}^+$ no es un grupo libre.
\end{example}

\marginnote[-2.2cm]{
\begin{kaobox}[frametitle=Aplicaciones nulas casi por todas partes]
	Sea $A$ un conjunto no vacío. Decimos que una aplicación $f\colon A
	\to \mb{Z}$ es \textbf{nula casi por todas partes} si podemos hallar un
	subconjunto $B \subseteq A$ finito tal que $f(x)=0$ para todo
	$x \in A\backslash B$.
\end{kaobox}
}

\begin{proposition}
	Sea $A \neq\emptyset$.
	Se define el grupo libre generado por $A$ como la familia $\mc{F}(A)$ de todas las aplicaciones $f\colon A \to \mb{Z}$ que se hacen cero casi por todas partes.
	Entonces, $\mc{F}(A)$ es un grupo libre dotado con la suma elemento a elemento.
>>>>>>> f4da141 (removing conflicts 2)
\end{proposition}

\begin{proof}
	Dado un $a \in A$, denotamos $\mc{X}_a$ como la aplicación característica del conjunto $\{a\}$.
	Claramente, $\mc{X}_a \in \mc{F}(A)$.
	También denotamos $\mc{X}$ como la familia de todos los $\mc{X}_a$ con $a \in A$.

	Dada $f \in \mc{F}(A)$, se define $\tilde{f}$ como
		\[\tilde{f}=\sum_{a \in A}f(a)\mc{X}_a\]
	Dado un $x \in A$, $\tilde f(x)=f(x)$, por lo que $\mc{X}$ es un sistema generador de $\mc{F}(A)$.
	
	Sean $\mu_1,\dots,\mu_n$ enteros y $a_1,\dots,a_n \in A$ tales que $\mu_1\mc{X}_{a_1}+\dots+\mu_n\mc{X}_{a_n}=0$.
	Dado un $1 \leq j \leq n$,
		\[0=\sum^n_{i=1}\mu_i\mc{X}_{a_i}(a_j)=\mu_j\mc{X}_{a_j}(a_j)=\mu_j\]
	por lo que $\mu_1,\dots,\mu_n=0$ y $\mc{X}$ es una base.
\end{proof}

\begin{remark}
	Dado un conjunto finito $A$, $\mc{F}(A)\cong \mb{Z}^{|A|}$.
\end{remark}

\begin{definition}
	Un \textbf{grupo graduado} es una colección de grupos $G=\{G_n: n \in \mb{Z}\}$.
	Un grupo graduado $H$ es subgrupo graduado (resp. normal) de $G$ si $H_j \subseteq G_j$ (resp. $H_j \trianglelefteq G_j$) para todo $j$.
	Si $H$ es un subgrupo graduado normal de $G$, definimos
		\[\frac{G}{H}:=\left\{\frac{G_i}{H_i}: i \in \mb{Z}\right\}\]
\end{definition}

Sean $G,H$ grupos graduados.
Un \textbf{homomorfismo graduado} $f\colon G \to H$ es una colección de homomorfismos $f_i\colon G_i \to H_{i+r}$, donde $r \in \mb{Z}$ es un valor común para todos los $f_i$, llamado \emph{grado} de $f$.
Diremos que $f$ es un monomorfismo (resp. epimorfismo, isomorfismo, endomorfismo, automorfismo) graduado si lo es cada uno de los homomorfismos en la colección.

\begin{definition}
	Un \textbf{complejo de cadenas} es un par de la forma $(G,\p)$, siendo $G$ un grupo graduado y $\p$ un endomorfismo de grado $-1$ tal que
		\[\im  \p_n \leq \ker \p_{n-1}\]
	La aplicación $\p$ recibe el nombre de \textbf{operador borde}.
\end{definition}

\begin{definition}
Sean $(C,d)$ y $(C',d')$ dos complejos de cadenas.
Una \textbf{aplicación de cadenas} es un homomorfismo graduado $\Phi\colon C \to C'$ de grado $0$ tal que el siguiente diagrama es conmutativo:
	\begin{diagram*}
	C_{n+1} \arrow{r}{d_{n+1}} \arrow{d}{\Phi_{n+1}} & C_n \arrow{d}{\Phi_n} \\
	C'_{n+1} \arrow{r}{d'_{n+1}}                     & C'_n                   
	\end{diagram*}
\end{definition}

Sea $(C,d)$ un complejo de cadenas.
Se definen los grupos graduados
	\begin{align*}
		Z_*(C)&:=\ker d=\{\ker d_n: n \in \mb{Z}\};	& Z_n(C)&:=\ker d_n\\[0.2cm]
		B_*(C)&:=\im d=\{\im d_n: n \in \mb{Z}\}; 	& B_n(C)&:=\im d_n
	\end{align*}
Diremos que dos elementos $p,q \in Z_n(C)$ son \textbf{homólogos} si $p-q \in B_n(C)$.

\begin{definition}
	Se define el \textbf{grupo graduado de homología} $C$ como 
		\begin{align*}
			H_*(C):=\frac{Z_*(C)}{B_*(C)}
		\end{align*}
	Diremos que dos complejos de cadenas tienen el mismo \textbf{tipo de homología} si sus grupos de homología son isomorfos.
\end{definition}

<<<<<<< HEAD
Sea $f\colon (C,d) \to (D,\p)$ una aplicación de cadenas.
Dado un $c \in Z_n(C)$,
	\[d_n(c)=0 \implies \p_n[f_n(c)]=f_{n-1}[d_n(c)]=f_{n-1}(0)\]
Como $f_{n-1}$ es un homomorfismo de grupos, $f_{n-1}(0)=0$, por lo que $f_n(c)\in Z_n(D)$ y $f_n[Z_n(C)]\leq Z_n(D)$.
Análogamente, $f_n(B_n(C)) \leq B_n(D)$.
Por tanto, $f$ induce un homomorfismo entre grupos de homología,
	\[f_*\colon H_*(C) \to H_*(D)\]
=======
Sea $(C,d)$ un complejo de cadenas. Se definen los grupos graduados
\begin{align*}
Z_*(C)&:=\ker d=\{\ker d_n: n \in \mb{Z}\};	& Z_n(C)&:=\ker d_n\\[0.2cm]
B_*(C)&:=\im d=\{\im d_n: n \in \mb{Z}\}; 	& B_n(C)&:=\im d_n
\end{align*}
Diremos que dos elementos $p,q \in Z_n(C)$ son \textbf{homólogos} si
$p-q \in B_n(C)$.
>>>>>>> f4da141 (removing conflicts 2)

\subsection{Homología de un espacio topológico}
\begin{definition}
<<<<<<< HEAD
	Se define el grupo de \textbf{$n$-cadenas singulares} $S_n(X)$ como el grupo libre generado por todos los símplices singulares $\phi\colon \sigma_n \to X$.
	Los elementos de $S_n(X)$ reciben el nombre de $n$-\emph{cadenas singulares} de $X$.
=======
Sea $C$ un complejo de cadenas. Se definen el \textbf{grupo graduado de
homología} y el \textbf{grupo de homología} de orden $n$ de $C$ como 
\begin{align*}
H_*(C):=\frac{Z_*(C)}{B_*(C)}; && H_n(C)=\frac{Z_n(C)}{B_n(C)}
\end{align*}
Diremos que dos complejos de cadenas tienen el mismo \textbf{tipo de homología}
si sus grupos de homología son isomorfos.
\end{definition}

Sea $f\colon (C,d) \to (D,\p)$ una aplicación de cadenas. Dado un
$c \in Z_n(C)$,
\[d_n(c)=0 \implies \p_n[f_n(c)]=f_{n-1}[d_n(c)]=f_{n-1}(0)\]
Como $f_{n-1}$ es un homomorfismo de grupos, $f_{n-1}(0)=0$, por lo que
$f_n(c)\in Z_n(D)$ y $f_n[Z_n(C)]\leq Z_n(D)$.

Análogamente, $f_n(B_n(C)) \leq B_n(D)$. Por tanto, $f$ induce un homomorfismo
entre grupos de homología,
\[f_*\colon H_*(C) \to H_*(D)\]

\subsection{Grupos de homología}
\begin{definition}
Se define el grupo de \textbf{$n$-cadenas
singulares} $S_n(X)$ como el grupo libre generado por todos los símplices
singulares $\phi\colon \sigma_n \to X$. Los elementos de $S_n(X)$
reciben el nombre de $n$-\emph{cadenas singulares} de $X$.
>>>>>>> f4da141 (removing conflicts 2)
\end{definition}

A partir los grupos de cadenas singulares, podemos definir el grupo graduado
	\[S_*(X)=\{S_n(X): n \geq 0\}\]
Para poder completar el grupo graduado, simplemente se considera que $S_p(X)=0$ para todo $p < 0$.

El objetivo de esta sección es definir un operador borde sobre $S_*(X)$, de forma que $(S_*(X),\p)$ forme una complejo de cadenas.
Podemos extender el operador cara a $S_n(X)$ tomando
	\[\p_{(j)}\left(\sum_{i=1}^n k_i\phi_i\right)= \sum^n_{i=1}k_i\,\p_{(j)}\phi_i\]
Este proceso da lugar a $n+1$ operadores cara diferentes, pero ninguno de ellos define un operador borde.

\begin{definition}
	Se define el \textbf{operador borde} $\p\colon S_n(X) \to S_{n-1}(X)$ asociado a $S_*(X)$ como
		\[\p=\p_{(0)}-\p_{(1)}+\dots+(-1)^n\p_{(n)}\]
\end{definition}

\begin{theorem}\labthm{OperadorBorde}
Dado un espacio topológico $X$, $\im \p \leq \ker \p$.
Esto implica que $(S_*(X), \p)$ es un complejo de cadenas.
\end{theorem}

\begin{proof}
Sea $c \in S_n(X)$.
Dado que $S_n(X)$ es un grupo libre y $\p$ es un homomorfismo, podemos suponer sin pérdida de generalidad que $c$ es un símplice singular $\sigma_n \to X$.

Dados $0 \leq p, q \leq n$ con $p < q-1$,
\begin{align*}
\p_{(p)}\p_{(q)}&c(t_0,\dots,t_{n-2})
	=\p_{(q)}c(t_0,\dots,t_{p-1},\hat t_p,t_p,\dots,t_{n-2})=\\
	&=c(t_0,\dots,t_{p-1},\hat t_p,t_p,\dots,t_{q-2},\hat t_q,t_{q-1},\dots,t_{n-2})=\\
	&=\p_{(p)}c(t_0,\dots,t_{q-2},\hat t_{q-1},t_{q-1},\dots,t_{n-2})=\\
	&=\p_{(q-1)}\p_{(p)}c(t_0,\dots,t_{n-2})
\end{align*}
donde $\hat t$ denota reemplazar $t$ con un cero.
Por otro lado,
\begin{equation}
	\label{Borde2c}
	\p^2c=\sum^{n-1}_{p=0}(-1)^p\p_{(p)}(\p c)=\sum^n_{q=0}\sum^{n-1}_{p=0}(-1)^{p+q}\p_{(p)}\p_{(q)}c
\end{equation}

Combinando ambas expresiones,
	\[\p^2c=\sum_{q=1}^n\sum_{p < q-1}(-1)^{p+q}\p_{(p)}\p_{(q)}c+\sum^n_{q=0}\sum_{p > q}(-1)^{p+q}\p_{(p)}\p_{(q)}c\]
\begin{marginfigure}
	\resizebox{\textwidth}{!}{
\begin{tikzpicture}[scale=.5]
%Región p > q
\filldraw[draw opacity=0, fill=red!30] (0,0) -- (5,5) -- (6,5) -- (6,0) -- cycle;
\draw[dashed] (0,0) -- (5,5);

\draw (0,5.5) -- (0,0) -- (6.5,0);
\draw (7,0) node {$q$};
\draw (0,6) node {$p$};

%q axis
\foreach \x in {0,...,4}
{
\draw (\x,-0.125) -- (\x,0.125);
\draw (\x,-0.5) node {$\x$};
}

\draw (5,-.5) node {$\dots$};

\draw (6,-0.125) -- (6,0.125);
\draw (6,-.5) node {$n$};

%p axis
\foreach \x in {0,...,3}
{
\draw (-0.125,\x) -- (0.125,\x);

\draw (-.5,\x) node {$\x$};
}

\draw (-.5,4) node {$\vdots$};

\draw (-0.125,5) -- (0.125,5);
\draw (-1,5) node {$n-1$};

\draw (4,2) node {$p > q$};
\end{tikzpicture}
}\labfig{borde2cero}
	\caption[Gráfica auxiliar que ilustra el cambio de índices.]{Gráfica auxiliar para visualizar el cambio de índices descrito en la ecuación \eqref{CambioIndices}.}
\end{marginfigure}

Obseremos que
\begin{multline} \label{CambioIndices}
	\{(p,q)\colon q=0,\dots,n,\, 0\leq p<q\}=\\
	\{(q,p)\colon p=0,\dots,n-1,\, p < q \leq n\}
\end{multline}
(ver \reffig{borde2cero}), por lo que $\p^2c$ se puede escribir como
\begin{align*}
	&\sum_{q=1}^n\sum_{p < q-1}(-1)^{p+q}\p_{(q-1)}\p_{(p)}c+\sum^n_{q=0}\sum_{p < q}(-1)^{p+q}\p_{(p)}\p_{(q)}c=\\ 
	&=\sum_{q=0}^{n-1}\sum_{p < q}(-1)^{p+q+1}\p_{(q)}\p_{(p)}c+\sum^{n-1}_{p=0}\sum_{q > p}(-1)^{p+q}\p_{(q)}\p_{(p)}c=\\
	&=\sum_{q=0}^{n-1}\left(-\sum_{p < q}(-1)^{p+q}\p_{(q)}\p_{(p)}c+\sum_{p < q}(-1)^{p+q}\p_{(q)}\p_{(p)}c\right)=0
\end{align*}
\end{proof}

Dado que $S_*(X)$ es abeliano por definición, su grupo de homología asociado está bien definido.
Además, como todo espacio topológico $X$ define un grupo graduado $S_*(X)$ de forma única, podemos introducir la siguiente notación sin ambigüedades:
\begin{align*}
	Z_n(X):=Z_n(S_*(X)); && B_n(X):=B_n(S_*(X))
\end{align*}

\begin{definition}
Sea $X$ un espacio topológico.
Se define el \textbf{grupo de homología de orden $n$ asociado a $X$} como
	\[H_n(X):=H_n(S_*(X))=\frac{Z_n(X)}{B_n(X)}\]
\end{definition}

\begin{example}\labexample{Camino1ciclo}
Sea $f\colon [0,1] \to X$ un camino. Se define el 1-símplice singular
\begin{funcion}
\psi\colon \sigma_1 \arrow[r] & X\\
(t_0,t_1) \arrow[r,maps to] & f(t_0)
\end{funcion}
Se tiene entonces que $\psi$ es un 1-ciclo si y sólo si $\psi(0,1)=\p_{(0)}\psi=\p_{(1)}\psi=\psi(1,0)$.
Dado que $\psi(0,1)=f(0)$ y $\psi(1,0)=f(1)$, un camino es un 1-ciclo si y sólo si $f$ es un lazo.
\end{example}

Sea $f\colon X \to Y$ una aplicación continua.
Habíamos definido una aplicación $f_\#$ que convierte símplices de $X$ en símplices de $Y$, al igual que el operador cara convertía $n$-símplices en $(n-1)$-símplices.
Podemos extender $f_\#$ a todo el grupo de cadenas singulares de forma que la aplicación resultante sea un homomorfismo:
<<<<<<< HEAD
	\[f_\#(n_1\phi_1+\dots+n_k\phi_k)=n_1f_\#(\phi_1)+\dots+n_kf_\#(\phi_k)\]
Esta aplicación recibe el nombre de \textbf{morfismo inducido en homología} por $f$.

\begin{proposition}\label{homo cadenas homologia}
	Todo morfismo inducido por una aplicación continua es una aplicación de cadenas.
	Como consecuencia, si $f\colon X \to Y$ es continua, $f_\#$ induce una familia de homomorfismos
	\begin{funcion*}
		f_*: H_n(X) \arrow[r] &H_n(Y)\\
		\left[x\right] \arrow[maps to,r]    &\left[f_\#(x)\right]
	\end{funcion*}
\end{proposition}

Dadas $f\colon X \to Y$ y $g\colon Y \to Z$ continuas, deducimos de las proposiciones \ref{ComposicionAlmohadilla} y \ref{homo cadenas homologia} que $(g\circ f)_*=g_*\circ f_*$.

\begin{theorem}\labthm{homologia invariante}
	Si $f\colon X \to Y$ es un homeomorfismo, $f_*\colon H_k(X) \to H_k(Y)$ es un isomorfismo para todo $k \geq 0$.
\end{theorem}

\subsection{Interpretación geométrica}
Sea $X=\mb{R}^2\backslash\{(0,0)\}$ con la topología inducida por $\mb{R}^2$.
El espacio $\mb{R}^2$ no es homeomorfo a $X$; sin embargo, no es posible probar este resultado utilizando sólo topología conjuntista.
=======
	\[f_\#\left(\sum^k_{j=1}n_j\phi_j\right)=\sum^k_{j=1}n_jf_\#(\phi_j)\]
Esta aplicación recibe el nombre de \textbf{morfismo inducido en homología} por $f$.

\begin{proposition}
	Todo morfismo inducido por una aplicación continua es una aplicación de cadenas.
	Como consecuencia, si $f\colon X \to Y$ es continua, $f_\#$ induce una familia de homomorfismos
	\begin{funcion}
		f_*: H_n(X) \arrow[r] &H_n(Y)\\
		\left[x\right] \arrow[maps to,r]    &\left[f_\#(x)\right]
	\end{funcion}
	para $n \geq 0$.
\end{proposition}

Dadas $f\colon X \to Y$ y
$g\colon Y \to Z$ continuas, deducimos de \refprop{ComposicionAlmohadilla} y de este resultado que
	\[(g\circ f)_*=g_*\circ f_*\]
Por tanto, deducimos que los grupos de homología son invariantes topológicos:

\begin{theorem}
	Sean $X,Y$ espacios topológicos.
	Si $f\colon X \to Y$ es un homeomorfismo, $f_*\colon H_k(X) \to H_k(Y)$ es un isomorfismo para todo $k \geq 0$.
	Por tanto, el grupo de homología es un invariante topológico.
\end{theorem}

\subsection{Interpretación geométrica}
Sea $X=\mb{R}^2\backslash\{(0,0)\}$ con la topología inducida por $\mb{R}^2$. El
espacio $\mb{R}^2$ no es homeomorfo a $X$; sin embargo, no es posible probar
este resultado utilizando sólo topología conjuntista. En esta sección, veremos
cómo la teoría de homología nos permite probar que no existe un homeomorfismo
entre $\mb{R}^2$ y $X$.
>>>>>>> f4da141 (removing conflicts 2)

Definimos un \textbf{camino orientado} en un espacio topológico $X$ como una terna $(\gamma; A,B)$, siendo $\gamma\colon [0,1] \to X$ un camino con $\gamma(0)\neq \gamma(1)$ y $A,B \in \gamma(\{0,1\})$ con $A\neq B$.
Diremos que $(\gamma; A,B)$ está \textbf{orientado positivamente} (resp. \textbf{negativamente}) si $A=\gamma(0)$ y $B=\gamma(1)$ (resp. $A=\gamma(1)$ y $B=\gamma(0)$).

\begin{example}
Sea $\eta\colon [0,1] \to \mb{R}^2$ el camino dado por
	\[\eta(t)=(\sin(\pi t),-\cos(\pi t))\]
La orientación negativa de $\eta$ (que podemos ver en \reffig{MotivHomologia}) viene dada por $A=\eta(1)=(0,1)$ y $B=\eta(0)=(0,-1)$.
\end{example}

\marginnote[-2.2cm]{
\begin{kaobox}[frametitle=Caminos compatibles]
	Diremos que dos caminos orientados $(\alpha;A_1,B_1)$ y $(\beta;A_2,B_2)$ en $X$ son \textbf{compatibles} si el camino $\gamma\colon [0,1] \to X$ dado por
		\[\gamma(t)=
			\begin{cases}
			\alpha(2t) & 0 \leq t \leq \frac{1}{2}\\
			\beta(2t-1) & \frac{1}{2} \leq t \leq 1
		\end{cases}\]
	es continuo y $B_1=A_2$.
\end{kaobox}
}

Consideremos los siguientes caminos positivamente orientados en $\mb{R}^2$:
\begin{align*}
	\eta(t)	&=(\sin(\pi t),-\cos(\pi t)); &&\phi(t)=(\cos(\pi t),\sin(\pi t));\\
	\mu(t)	&=(2\sin(\pi t),-\cos(\pi t));
\end{align*}

Los caminos $\phi$ y $\eta$ tienen orientaciones compatibles, y forman una circunferencia cuyo interior está contenido en $\mb{R}^2$.
Por tanto, podemos hallar una cadena singular cuyo borde sea $\phi+\eta$.
Si tomamos clases módulo $B_1(\mb{R}^2)$,
	\[\phi+\eta \in B_1(\mb{R}^2) \iff -[\eta]=[\phi]\]

Los caminos $\eta$ y $\mu$ forman una figura homotópica a una circunferencia, y sus orientaciones son compatibles.
Podemos encontrar una cadena singular cuyo borde sea $\phi+\mu$. Tomando clases,
	\[\phi+\mu \in B_1(\mb{R}^2) \iff [\phi]=-[\mu]\]

\begin{marginfigure}
	\resizebox{\textwidth}{!}{
\begin{tikzpicture}
\draw[thick] (0,0) circle (2cm);
\draw (0,0) node {$\star$};

\draw[fill=black] (0,2) circle (1.5pt);
\draw (0,2.5) node {$A$};

\draw [-stealth] (2,0) -- (2,1);
\draw (2.5,0.5) node {$\eta$};

\draw[fill=black] (0,-2) circle (1.5pt);
\draw (0,-2.5) node {$B$};

\draw [-stealth] (-2,0) -- (-2,-1);
\draw (-2.5,-0.5) node {$\phi$};

%Primera distancia -> Eje horizontal
%Segunda distancia -> Eje vertical
\draw[thick] (0,2) arc (90:-90:4cm and 2cm);

\draw [-stealth] (4,0) -- (4,-1);
\draw (4.5,-0.5) node {$\mu$};
\end{tikzpicture}
}
	\caption{Varios caminos en $\mb{R}^2$.\labfig{MotivHomologia}}
\end{marginfigure}

Consieremos ahora el plano perforado, $X=\mb{R}^2\backslash\{(0,0)\}$.
La aplicación $\phi|_X+\eta|_X$ ya no forma el borde de una cadena singular, dado que los caminos encierran al punto $(0,0)$, que no está.
Por tanto, las clases de $\phi|_X$ y $\mu|_X$ serán diferentes.
En cambio, $\phi$ y $\psi$ siguen siendo homólogas, porque sus orientaciones son compatibles y no contienen al punto que hemos quitado.

Como consecuencia del \refthm{homologia invariante}, concluimos que $\mb{R}^2$ no es homeomorfo a $X$.

\subsection{Característica de Euler}
<<<<<<< HEAD
\marginnote[-2.2cm]{
\begin{kaobox}[frametitle=Rango de un grupo]
	Sea $A$ un grupo abeliano.
	Se denomina \textbf{subgrupo de torsión} de $A$ al subgrupo $T$ formado por todos los elementos de orden finito de $A$.
	Decimos que $A$ es \textbf{libre de torsión} si $T=0$, y que $A$ es un \textbf{grupo de torsión} si $T=A$.
	Definimos el \textbf{rango} de $A$ como el mínimo número de generadores que posee su subgrupo de torsión.
\end{kaobox}
}
=======
\begin{definition}
Sea $A$ un grupo abeliano. Se denomina \textbf{subgrupo de torsión} de $A$ al
subgrupo $T$ formado por todos los elementos de orden finito de $A$. Decimos que
$A$ es \textbf{libre de torsión} si $T=0$, y que $A$ es un \textbf{grupo de
torsión} si $T=A$.
\end{definition}

Si $T$ es el subgrupo de torsión de un cierto grupo abeliano $A$,
$A/T$ es un grupo libre de torsión.
>>>>>>> f4da141 (removing conflicts 2)

\begin{example}
\begin{enumerate}
	\item El grupo aditivo $\mb{Z}_n=\mb{Z}/n\mb{Z}$ es un grupo de torsión: dado un $\overline{p} \in \mb{Z}_n$,
		\[n\overline{p}=\sum^{n|p|}_{i=1}\overline{1}=\sum^{|p|}_{i=1}\overline{n}=\sum^{|p|}_{i=1}0=0\]
	Por la misma razón, todo cuerpo de característica mayor que cero define un grupo de torsión.
	\item El grupo aditivo $\mb{Z}$ es un grupo libre de torsión porque no tiene divisores de cero: si $n > 0$ y $p \in \mb{Z}$ es tal que $np=0$, necesariamente se cumple que $p=0$.
	\item Consideramos $\mb{Z}\times \mb{Z}_n$: dado un $\overline{q} \in \mb{Z}_n$,
		\[n(0,\overline{q})=\sum^n_{i=1}(0,\overline{q})=(0,n\overline{q})=(0,0)\]
	por lo que el subgrupo de torsión de $\mb{Z}\times \mb{Z}_n$ contiene a $\{0\}\times\mb{Z}_n$.
	Sin embargo, si $m > 0$ y $p \in \mb{Z}$,
		\[m(p,\overline{q})=\sum^m_{i=1}(p,\overline{q})=(mp,m\overline{q})=(0,0)\]
	luego $n$ divide a $mq$ y $p=0$.
	Por tanto, el subgrupo de torsión de $\mb{Z}\times \mb{Z}_n$ es $\{0\}\times\mb{Z}_n$ (que es un subgrupo propio).
\end{enumerate}
\end{example}

\begin{definition}
Se define el \textbf{$n$-ésimo número de Betti} $\beta_n(X)$ como el rango de $H_n(X)$.
Si existe un $k \in \mb{N}$ tal que $\beta_p(X)=0$ para todo $p > k$, se define la \textbf{característica de Euler} de $X$ como 
	\[\chi(X):=\sum^k_{n=0}(-1)^n\beta_n(X)\]
\end{definition}

%\setchapterpreamble[u]{\margintoc}

\section{Homología de un espacio arcoconexo}
Sea $X$ un espacio topológico arcoconexo.
Dados dos puntos $x,y \in X$, existe un camino $L_{x,y}\colon [0,1] \to X$ tal que $L_{x,y}(0)=x$ y $L_{x,y}(1)=y$.
Como vimos en \refexample{Camino1ciclo}, $L_{x,y}$ induce un $1$-símplice singular $\phi\colon \sigma_1 \to X$.
En particular, $L_{x,x}=\cte_x$ es una aplicación constante que queda totalmente determinada por $x$, por lo que podemos identificar $x$ con $L_{x,y}$.

Dado que $\cte_x \in S_0(X)$, existirán $x_1,\dots,x_n \in X$ y enteros $\mu_1,\dots,\mu_n$ tales que
\[x \equiv \cte_x=\sum^n_{i=1}\mu_i\cte_{x_i} \equiv \sum^n_{i=1}\mu_ix_i\]

Consideremos el último tramo del complejo de cadenas $S_*(X)$:
\begin{diagram*}
	S_1(X) \arrow{r}{\p_1} & S_0(X) \arrow{r}{\p_0} & 0
\end{diagram*}
Dado que $\p_0=0$, $S_0(X)=Z_0(X)$ y podemos construir el siguiente homomorfismo entre $S_0(X)$ y $\mb{Z}$:
\begin{funcion*}
	\beta\colon S_0(X) \arrow{r}           & \mb{Z}          \\
	n_1x_1+\dots+n_px_p \arrow[r, maps to] & n_1+\dots+n_p
\end{funcion*}
Como $X\neq\emptyset$, $\beta$ es epimorfismo, ya que $p=\beta(px)$ para todo $x \in X$.

Sea $\phi \in S_1(X)$: como $\p$ es un homomorfismo de grado $-1$, $\p\phi \in S_0(X)$.
En particular, $\phi$ es una aplicación continua que depende de dos variables no negativas cuya suma siempre es 1, luego
	\[\p\phi=\phi(0,t_0)-\phi(t_0,0)=\phi(0,1)-\phi(1,0)\]
y $\p \phi\in \ker \beta$.
Si $c=n_1\phi_1+\dots+n_p\phi_p$,
	\[\beta(\p c)=
		\beta(n_1\p\phi_1+\dots+n_p\phi_p)=
		n_1\beta(\p\phi_1)+\dots+n_p\beta(\p\phi_p)=0\]
Como $\p c$ son los elementos de $B_0(X)$, $B_0(X) \subseteq\ker \beta$.

Recíprocamente, sea $c \in \ker \beta$: identificando $x \in X$ con $\cte_x$, existirán $x_1,\dots,x_k \in X$ tales que $c=n_1x_1+\dots+n_kx_k$.
Como $X$ es arcoconexo, dado un $x \in X$, podemos construir la cadena singular $d=n_1L_{x,x_1}+\dots+n_kL_{x,x_k}$ y
	\[\p d=
		\sum^p_{i=1}n_i(L_{x,x_i}(1)-L_{x,x_i}(0))=
		\sum^k_{i=1}n_ix_i-x\sum^k_{i=1}n_i.\]
Como $c \in \ker \beta$, se sigue que
	\[\sum^p_{i=1}n_i=\beta(c)=0 \implies \p d=\sum^p_{i=1}n_ix_i=c.\]

Por tanto, $c \in B_0(X)$ y $B_0(X)=\ker \beta$.
De aquí se sigue que
	\[H_0(X)=\frac{Z_0(X)}{B_0(X)}=\frac{S_0(X)}{\ker \beta} \cong \mb{Z}\]

\begin{proposition}
	Si $X$ es arcoconexo, $H_0(X)$ tiene rango $1$.
\end{proposition}

\begin{example}
	Sea $X=\{\star\}$.
	Dado un $n \geq 0$, existe un único $n$-símplice singular $\phi_n\colon\sigma_n \to X$, que es la aplicación constante.
	Las caras de la aplicación constante son la aplicación constante, por lo que
	\begin{align*}
	\p\phi_n=
		\begin{cases}
		\phi_{n-1}	&\mbox{ si }n\mbox{ es par}\\
		0          	&\mbox{ si }n\mbox{ es impar}
		\end{cases}
	\end{align*}

	Dado que $S_n(X)=\la \phi_n\ra$, el operador borde $\p_n\colon S_n(X) \to S_{n-1}(X)$ es un isomorfismo si $n$ es par y cero si $n$ es impar.
	Por tanto, $H_n(X)=0$ para todo $n > 0$.
\end{example}

\subsection{Sumas directas en homología}
\begin{definition}
	Sea $\{G_\alpha\}_{\alpha \in A}$ una familia de grupos abelianos.
	Se define la \textbf{suma directa} de los grupos $G_\alpha$ como el grupo $G$ de aplicaciones $f\colon A \to G$ con soporte finito.

	Si $A$ es finito, escribiremos $G=G_{\alpha_1}\oplus \dots \oplus G_{\alpha_k}$.
\end{definition}

Denotaremos a las aplicaciones $f \in G$ como tuplas de la forma $(f_\alpha)_{\alpha \in A}=\{f(\alpha)\}_{\alpha \in A}$.
Los elementos $f_\alpha$ reciben el nombre de \textbf{componentes} de $f$.

Dada una familia de complejos de cadenas $\{C^\alpha\}_{\alpha \in A}$, se define el grupo graduado
	\[C=\sum_{\alpha \in A}C^\alpha:=\left\{\sum_{\alpha \in A}C^\alpha_p: p \in \mb{Z}\right\}\]
donde $\sum$ denota suma directa.

Dado un $c \in C_p$, existen $\alpha_1,\dots,\alpha_n \in A$ tales que $c(\alpha)=0$ para todo $\alpha \in A\backslash \{\alpha_1,\dots,\alpha_n\}$.
Podemos entonces identificar $c$ con un elemento $(c^{\alpha_1},\dots,c^{\alpha_n}) \in C^{\alpha_1}_p\oplus\dots\oplus C^{\alpha_n}_p$.
Si denotamos al operador borde de $C^{\alpha_j}$ como $\p^{\alpha_j}$,
	\[\left(\p^{\alpha_1}_pc^{\alpha_1},\dots,\p^{\alpha_n}_pc^{\alpha_n}\right) \in C^{\alpha_1}_{p-1}\oplus\dots\oplus C^{\alpha_n}_{p-1}\]
Por tanto, definimos la siguiente aplicación:
	\begin{funcion*}
	\p_p\colon C_p \arrow[r]             & C_{p-1}                   \\
	c \arrow[r, maps to] &
	(\p^{\alpha_1}_pc^{\alpha_1}_p,\dots,\p^{\alpha_n}c^{\alpha_n}_p)
	\end{funcion*}
Dado que $\p_p$ actúa componente a componente, es inmediato que esta construcción da lugar a un operador borde.

\begin{lemma}\lablemma{HomoSumaDir}
	Dada una familia de complejos de cadenas $\{C^\alpha\}_{\alpha \in A}$ con suma directa $C$,
	\[H_*(C)\cong\sum_{\alpha \in A}H_*(C^\alpha)\]
\end{lemma}

\subsection{Descomposición en arcocomponentes}
Supongamos $\Omega$ es un espacio topológico formado por dos arcocomponentes, $\Omega_1$ y $\Omega_2$, como los que se muestran en \reffig{FigPentaHexa}.
Si $f(x) \in \Omega_1$, por continuidad, $f(\sigma_p) \subseteq \Omega_1$.
De esta forma, todos los elementos de la base de $H_p(\Omega)$ están en $H_p(\Omega_1)$ o en $H_p(\Omega_2)$; es decir,
\[H_p(\Omega)\cong H_p(\Omega_1)\oplus H_p(\Omega_2)\]

\begin{marginfigure}
	\resizebox{\textwidth}{!}{
\begin{tikzpicture}
\draw[fill=blue, fill opacity=0.5] (-1,-1) circle (1cm);
\draw (-2,0) node {$\Omega_1$};

\draw[thick] (-1,-1) arc (15:114:0.5);

\draw[fill=green, fill opacity=0.5] (1,1) circle (1cm);
\draw (0,2) node {$\Omega_2$};

\draw[thick] (1,1) arc (0:95:0.5);

\draw[thick, dashed] (-1,-1) -- (1,1);
\end{tikzpicture}
}
	\caption[Dos espacios arcoconexos.]{\labfig{FigPentaHexa}Los arcos de circunferencia son $1$-símplices singulares de $\Omega$, pero la línea discontinua uniéndolos no lo es, ya que se sale del espacio.}
\end{marginfigure}

\begin{proposition}
	\labprop{SumaDirArco}
	Sea $X$ un espacio topológico y $\{X_\alpha\colon \alpha\in A\}$ la familia de arcocomponentes de $X$.
		\[H_*(X) \cong \sum_{\alpha \in A} H_*(X_\alpha)\]
\end{proposition}

\begin{proof}
	Consideramos la aplicación
	\begin{funcion*}
		\Psi_k\colon\sum_\alpha S_k(X_\alpha) \arrow[r] & S_k(X)          \\
		(g_\alpha)_\alpha \arrow[r, maps to]            & \sum_\alpha g_\alpha
	\end{funcion*}
	Toda cadena singular $g \in S_k(X)$ admite una descomposición única como combinación lineal de $k$-símplices singulares.
	Dado que cada símplice va a parar a una única componente conexa de $X$, se sigue que $\Psi_k(g)=\Psi_k(h)$ implica $g=h$, por lo que $\Psi_k$ es inyectiva.

	Sea $\phi\colon \sigma_k \to X$ un $k$-símplice singular.
	Las aplicaciones continuas preservan la arcoconexión, por lo que $\phi(\sigma_k)$ está en alguna componente arcoconexa $X_\alpha \subseteq X$.
	Por tanto, podemos hallar un $\phi_\alpha \in S_k(X_\alpha)$ tal que $\Psi_k(\phi_\alpha)=\phi$ y $\Psi_k$ es sobreyectiva.

	Dado $g=(g_\alpha)_\alpha$,
		\[\Psi_k(\p g)=\Psi_k({\p^\alpha g_\alpha})=\sum_{\alpha \in A}\p^\alpha g_\alpha=\p \sum_{\alpha \in A}g_\alpha=\p \Psi_k(g)\]
	Se sigue que $\Psi$ es aplicación de cadenas y
		\[H_*(X)=H_*[S_*(X)]\cong H_*\left[\sum_{\alpha \in A}S_*(X_\alpha)\right]\]

	Finalmente, aplicando \reflemma{HomoSumaDir},
		\[H_*\left[\sum_{\alpha \in A}S_*(X_\alpha)\right]\cong\sum_{\alpha \in A}H_*[S_*(X_\alpha)]=\sum_{\alpha \in A}H_*\left(X_\alpha\right)\]
\end{proof}

Como consecuencia de este resultado, podemos asumir que todo espacio topológico es arcoconexo.

Sea $X_\alpha$ una arcocomponente de $X$ y $x,y \in Z_0(X_\alpha)$.
Si $a,b \in \mb{Z}$, $ax+by \in Z_0(X_\alpha)$.
Teniendo en cuenta que $ax+by=a(x-y)+y(a+b)$,
	\[[ax+by]=a[x-y]+(a+b)[y]=(a+b)[y] \implies H_0(X_\alpha)=\la [y]\ra\]
Si $X$ tiene $n$ arcocomponentes, existirán $y_i \in H_0(X_i)$ ($1 \leq i \leq n$) tales que
	\[H_0(X)=\sum^n_{\alpha=1}\la [y_\alpha]\ra\cong\sum^n_{i=1}\mb{Z} \cong \mb{Z}^n\]

\subsection{Homología de un conjunto convexo}
El objetivo de este apartado es probar el siguiente resultado:
\begin{theorem}\labthm{Convexo}
	Sea $X \subset \mb{R}^n$ un conjunto convexo. Dado un $p > 0$, $H_p(X)=0$.
\end{theorem}

\begin{lemma}\lablemma{ConvexoLema}
	Sea $X$ un conjunto convexo.
	Dado un símplice singular $\phi\colon \sigma_{p} \to X$, la aplicación $T_p(\phi)\colon \sigma_{p+1} \to X$ dada por
	\[(t_0,\dots,t_{p+1})\mapsto
		\begin{cases}
		\displaystyle(1-t_0)
		\phi\left(\frac{t_1}{1-t_0},\dots,\frac{t_{p+1}}{1-t_0}\right)+t_0x
		&\mbox{ si }t_0 < 1\\
		x & \mbox{ si }t_0=1
		\end{cases}
	\]
	es un símplice singular.
\end{lemma}

\begin{proof}
	Si $t_0 < 1$,
	\begin{equation}\label{BienDef}
		\sum^{p+1}_{j=0}t_j=
		1 \iff \sum^{p+1}_{j=1}t_j=
		1-t_0 \iff \sum^{p+1}_{j=1}\frac{t_j}{1-t_0}=1
	\end{equation}
	por lo que los puntos de la forma $\frac{t_j}{1-t_0}$ están en $\sigma_p$ y $T_p(\phi)$ está bien definida.

	Sea $\tau_j=\frac{t_j}{1-t_0}$.
	La aplicación $T_p(\phi)$ es continua cuando $t_0 < 1$.
	Si $t_0=1$,
	\begin{multline*}
		0	\leq \|T_p(\phi)(t_0,\dots,t_{p+1})-x\|=\|(1-t_0)\phi(\tau_1,\dots,\tau_{p+1})-(1-t_0)x\|\leq\\
			\leq (1-t_0)(\|\phi(\tau_1,\dots,\tau_{p+1})\|+\|x\|)
	\end{multline*}

	Dado que $\phi$ es continua y $\sigma_p$ es compacto, $\phi(\sigma_p)$ es un subconjunto compacto de $\mb{R}^n$, por lo que está acotado.
	Podemos encontrar un $M > 0$ tal que $\|\phi(\tau_1,\dots,\tau_{p+1})\|$, por lo que
		\[0 \leq \|T_p(\phi)(t_0,\dots,t_{p+1})-x\| \leq (1-t_0)(M+\|x\|)\]
	De estas desigualdades, concluimos que
		\[\lim_{t_0 \to 1}T_p(\phi)(t_0,\dots,t_{p+1})=x\]
	y $T_p(\phi)$ es un símplice singular.
\end{proof}

\begin{proof}[Demostración del teorema]
	Dado un $p$-símplice singular $\phi$, sea $T_p(\phi)$ el símplice definido en el lema \reflemma{ConvexoLema}.
	Queremos ver que
	\begin{equation}
		\phi=\p T_p(\phi)+T_p(\p \phi) \label{IdTp}
	\end{equation}

	Por un lado, $\p T_p(\phi)(t_0,\dots,t_p)=\phi(t_0,\dots,t_p)$, por lo que $\p\circ T_p$ es la identidad.
	Por otro lado, dado $i=1,\dots,p+1$,
	\begin{align*}
	T_p(\p_{(i-1)}\phi)(t_0,\dots,t_p)
		&=(1-t_0)(\p_{(i-1)}\phi)(\tau_1,\dots,\tau_p)+t_0x=\\
		&=(1-t_0)\,\phi(\tau_1,\dots,\tau_{i-1},0,\tau_{i-1},\dots,\tau_p)+t_0x=\\
		&=T_p(\phi)(t_0,\dots,t_{i-1},0,t_i,\dots,t_p)=\\
		&=\p_{(i)}T_p(\phi)(t_0,\dots,t_p)
	\end{align*}
	de forma que $\p_{(i)}(T_p\phi)=T_p(\p_{(i-1)}\phi)$.
	Combinando ambas identidades,
	\begin{align*}
	\p T(\phi)
		&=\p_{(0)}T(\phi)+\sum^{p+1}_{i=1}(-1)^i\partial_{(i)}(T_p\phi)=\\
		&=\phi+\sum^p_{i=0}(-1)^{i+1}T_p(\p_{(i)}\phi)=\phi-\sum^p_{i=0}(-1)^iT_p(\p_{(i)}\phi)
	\end{align*}
	de donde se sigue la igualdad deseada.

	Sea $z \in Z_p(X)$.
	Usando \eqref{IdTp}, $z=\p T_p(z)+T_p(\p z)=\p T_p(z)$, por lo que $z \in B_p(X)$ y $H_p(X)=0$.
\end{proof}

\section{Homotopías en el grupo de homología}\labsec{HomoHomo}
\begin{definition}
	Decimos que $X$ es \textbf{contráctil} si existe un punto $p \in X$ y una homotopía $F_p$ tal que
		\[F_p\colon \id_X \simeq \cte_p\]
\end{definition}

Sea $X$ un espacio contráctil. Por definición, existe un punto $p \in X$ y una
homotopía $F\colon X \times [0,1] \to X$ tales que $F\colon \id_X \simeq
\cte_p$. Si ahora consideramos dos puntos $x,y \in X$, podemos construir el
camino $\alpha\colon [0,1] \to X$ dado por
\[\alpha(t)=
\begin{cases}
F(x,2t) & \text{ si $t < 1/2$}\\
F(y,1-2t) & \text{ si $t \geq 1/2$}
\end{cases}\]

Observamos que $F(x,1)=p=F(y,1)$, de forma que $\alpha$ es continua en
$t=1/2$. Por tanto, podemos hallar un camino que conecta todo par de puntos en
$X$, de forma que $X$ es arcoconexo.

\begin{example}
Sea $A \subseteq \mb{R}^n$ un conjunto convexo. Por ser convexo, dado un $a \in
A$, el segmento $[0,a]$ está contenido en $A$. Eso nos permite definir la
aplicación continua
\begin{diagram}
F\colon A\times I \arrow[r] & A                      \\[-0.8cm]
(a,t) \arrow[r, maps to]      & (1-t)a
\end{diagram}
por lo que $F\colon \id_A \simeq \cte_0$
\end{example}

\begin{marginfigure}
\resizebox{\textwidth}{!}{
\begin{tikzpicture}
\draw (0,4) -- (0,0) -- (3.5,0);

\foreach \x in {1,...,15}
{
\draw (3.5/\x,4) -- (3.5/\x,0);
}

\draw (3.5,-.35) node {$(1,0)$};
\draw (3.5/2,-.35) node {$(\frac{1}{2},0)$};
\draw (3.5/4,-.35) node {$\hdots$};
\draw (0,-.35) node {$(0,0)$};

\draw (3.5,4.35) node {$(1,1)$};
\draw (3.5/2,4.35) node {$(\frac{1}{2},1)$};
\draw (3.5/4,4.35) node {$\hdots$};
\draw (0,4.25) node {$(0,1)$};
\end{tikzpicture}
}
\caption[Peine del topólogo.]{\labfig{Peine} Primeras $15$ iteraciones del
peine del topólogo. La iteración $w$ añade el segmento correspondiente a
$x=1/w$.}
\end{marginfigure}

\begin{example}
Sea $\Sh$ (pronunciado \emph{sh}) el peine del topólogo. Consideremos la
aplicación continua
\begin{diagram}
F\colon \Sh\times I \arrow[r]             & \Sh                   \\[-8mm]
(x,y,t) \arrow[r, maps to] & (x,(1-t)y)
\end{diagram}
y la proyección sobre el eje de abscisas, $\pi\colon \mb{R}^2 \to \mb{R}$. La
aplicación $F$ es una homotopía que baja las púas de $\Sh$:
\[F: \id_\Sh \simeq (\pi\times \cte_0)\]
Dado que $\pi \simeq \cte_{(1,0)}$, $\Sh$ es un espacio contráctil.
\end{example}

\begin{definition}
Un subespacio $A$ de $X$ es un \textbf{retracto débil} si podemos hallar una
aplicación continua $r\colon X \to A$ tal que $r|_A\simeq \id_A$. Si podemos
construir $r$ de forma que $r|_A=\id_A$, decimos que $A$ es un \textbf{retracto
fuerte} (o simplemente \textbf{retracto}) y $r$ es una \textbf{retracción}.
\end{definition}

Sea $A$ un retracto de $X$ y $r\colon X \to A$ una retracción. Por definición de
retracto, si $i\colon A \hookrightarrow X$ es la inclusión,
\[r\circ i=\id_A \implies \id_{H_n(A)}=(1_A)_*=(r\circ i)_*=r_*\circ i_*\]
por lo que $r_*$ es sobreyectiva e $i_*$ es inyectiva.

\begin{example}
Sea $D^2$ la bola de centro $p=(0,0)$ y radio 1 de $\mb{R}^2$. Se considera la
aplicación
\begin{diagram}
r\colon D^2-\{p\} \arrow[r]             & S^1                   \\[-8mm]
x \arrow[r, maps to] & \frac{x}{\|x\|}
\end{diagram}
La aplicación $r$ es continua por ser cociente de funciones continuas. Además,
$r|_{S^1}=\id_{S^1}$. Se sigue que $S^1$ es un retracto de $D^2-\{p\}$.
\end{example}

\begin{example}[\cite{Spanier66}, p. 28]
Sea $X=[0,1]\times[0,1]$ el cuadrado unidad en $\mb{R}^2$. La inclusión $i\colon
\Sh \hookrightarrow X$ define un retracto débil. Sin embargo, se puede probar que
$\Sh$ no es un retracto fuerte de $X$.
\end{example}

\begin{definition}
Sea $A$ un subespacio de $X$. Decimos que $X$ es \textbf{deformable} en $A$ si
existe una aplicación continua $r\colon X \to A$ llamada \textbf{deformación} tal
que $i\circ r \simeq \id_X$, siendo $i\colon A \hookrightarrow X$ la inclusión.
\end{definition}

Dado que $i\circ r\simeq \id_X$, $i_*$ es sobreyectiva y $r_*$ es inyectiva.

\begin{example}\labexample{Deformacion}
\begin{enumerate}
\item Sea $\pi\colon \mb{R}^2 \to \mb{R}$ la proyección sobre el eje de abscisas.
Tomando la aplicación $r(t)=(\cos(2\pi t),\sin(2\pi t))$, se tiene que $r\circ
\pi\colon D^2 \to S^1$ es una deformación. No obstante, si tomamos el punto
$(0,1) \in S^1$,
\[(r\circ \pi)(0,1)=r(0)=(\cos 0,\sin 0)=(1,0)\neq (0,1)\]
por lo que $r\circ \pi$ no es una retracción.
\item Sea $X$ un espacio contráctil a un cierto punto $q$. Si $A$ es un
subespacio de $X$ que contiene a $q$, se tiene que $\cte_q\colon X \to A$ es
una deformación de $X$ en $A$ por definición de espacio contráctil. Por tanto,
$X$ es deformable en cualquier subespacio que contenga a $q$.
\end{enumerate}
\end{example}

\begin{definition}
Decimos que un subespacio $A$ es un \textbf{retracto por deformación fuerte} de
$X$ si podemos hallar una homotopía $F\colon X\times I \to X$ tal que 
\begin{enumerate}
\item dado un $x \in X$, $F(x,0)=x$;
\item $F(X,1) \subseteq A$;
\item dado un $a \in A$ y un $t \in I$, $F(a,t)=a$.
\end{enumerate}
\end{definition}

En particular, observamos que un retracto por deformación fuerte describe
tanto una deformación (ítem 2) como una retracción (ítem 3).

\begin{example}
El peine del topólogo no es un retracto por deformación fuerte del cuadrado
unidad, ya que no es un retracto fuerte.
\end{example}

Sea $i\colon A \hookrightarrow X$ la inclusión. Si $A$ es un retracto por
deformación de $X$, $X$ es deformable en $A$ ($i_*$ es un epimorfismo), pero
$A$ es un retracto de $X$ ($i_*$ es un monomorfismo). Por tanto, se tiene el
siguiente corolario:

\begin{corollary}\label{RDFHomo}
Los retractos por deformación fuerte no alteran el tipo de homología de un
espacio topológico.
\end{corollary}

\subsection{Homotopías de cadenas}
Sean $C$ y $D$ complejos de cadenas, con $f,g\colon C \to D$ aplicaciones de
cadenas. Queremos hallar una condición suficiente para poder afirmar que $f$ y $g$
inducen el mismo homomorfismo entre $H_p(C)$ y $H_p(D)$. Una forma de comprobar
esto es verificar que la aplicación de cadenas $\alpha=f-g\colon C \to D$ induce
el homomorfismo nulo en homologías:
\[f_*=g_* \iff \alpha_*=f_*-g_*=0\]

Sea $c \in Z_p(C) \leq C_p$. Supongamos que existe un $b \in D_{p+1}$ tal que
$\alpha(c)=\p b$. Por cómo se define $B_p(D)$, es inmediato que
\[\alpha_*([c])=[\p b]=0+B_p(D)\]
Existirá una homomorfismo $S\colon C_p \to D_{p+1}$ tal que $\alpha=\p \circ S$.
Pero esta no es una aplicación de cadenas, por lo que no sabemos si va a inducir
un homomorfismo entre los grupos de homología.

Como $S$ es un homomorfismo, $S(0)=0$; así, si tomamos $T=S|_{Z_p(C)}$, se
verifica que $\alpha=\p \circ T + T \circ \p$. Veamos que esta nueva definición
de $\alpha$ conmuta con el operador borde:
\begin{align*}
\p \circ \alpha &=
\p^2 \circ T+\p \circ T \circ \p =
\p \circ T \circ \p=\\
&=\p \circ T \circ \p + T \circ \p^2 =
\alpha \circ \p
\end{align*}

Por construcción, $\alpha_*$ es el homomorfismo nulo. Pero $\alpha=f-g$, por lo
que hemos encontrado una condición suficiente para que $f$ y $g$ induzcan el
mismo homomorfismo entre grupos de homología.

\begin{definition}
Sean $C$ y $D$ complejos de cadenas. Dos aplicaciones de cadenas
$f,g\colon C \to D$ son \textbf{homotópicas} si existe un homomorfismo $T\colon
C \to D$ de grado 1 tal que
\[f-g=\p \circ T+T\circ \p\]
La aplicación $T$ recibe el nombre de \textbf{homotopía de cadenas} entre $f$ y
$g$.
\end{definition}

\subsection{Aplicaciones homotópicas}
\begin{theorem}[Teorema de invarianza homotópica de la homología]
Dadas dos aplicaciones continuas $f,g\colon X \to Y$ homotópicas, $f_*=g_*$.
\end{theorem}

\begin{proof}
Sea $F\colon X \times I \to Y$ una homotopía entre $f$ y $g$. Se definen las
aplicaciones $\alpha,\beta\colon X \to X\times I$ dadas por las expresiones
\begin{align*}
\alpha(x)=(x,0); && \beta(x)=(x,1)
\end{align*}
de forma que $f=F\circ \alpha$ y $g=F\circ \beta$.

Supongamos que existe una homotopía de cadenas $T\colon S_*(X) \to
S_*(X\times I)$ entre $\alpha$ y $\beta$. $T$ induce una homotopía de cadenas
entre $f_\#$ y $g_\#$ de la siguiente forma:
\begin{align*}
f_\#-g_\#&=F_\#\circ\alpha_\#-F_\#\circ\beta_\#=
	F_\#\circ(\alpha_\#-\beta_\#)=\\
	&=F_\#\circ(\p \circ T+T\circ\p)=
	(F_\#\circ\p)\circ T+(F_\#\circ T)\circ\p
\end{align*}
Como $F_\#$ es una aplicación de cadenas,
\begin{align*}
(F_\#\circ\p)\circ T+(F_\#\circ T)\circ\p&=
(\p\circ F_\#)\circ T+(F_\#\circ T)\circ\p=\\
&=\p\circ (F_\#\circ T)+(F_\#\circ T)\circ\p
\end{align*}
por lo que $f_*=g_*$, que es lo que queríamos probar. Por tanto, bastará con
probar que $T$ existe.

Sea $\tau_n\colon \sigma_n \to\sigma_n$ la aplicación identidad. Procedemos a
construir $T$ de forma inductiva: supongamos que $X=\sigma_0$. El símplice
$\sigma_0$ es el espacio puntual formado por el punto 1, de forma que definimos
la 0-cadena
\[c=\alpha_\#(\tau_0)-\beta_\#(\tau_0)=\alpha-\beta \in S_0(\sigma_0\times I)\]
Como $S_0(\sigma_0\times I)=Z_0(\sigma_0\times I)$, $c$ es un 0-ciclo.

Dado que $\sigma_0$ es un espacio puntual, podemos identificar $\alpha$ con
$\alpha(1)=(1,0)$ y $\beta$ con $\beta(1)=(1,1)$. Como $\sigma_0\times I$ es
arcoconexo, existe un camino
\[b\colon I \to \sigma_0\times I\]
tal que $b(0)=(1,0)\equiv \alpha$ y $b(1)=(1,1)\equiv\beta$. Se sigue que $c=
\p b$. Definimos entonces $T_{\sigma_0}(\tau_0):=b$.

Si $X$ es un espacio topológico arbitrario y $\phi\colon \sigma_0 \to X$ un
0-símplice singular, definimos
\[T_X(\phi):=(\phi \times \id_I)_\#(T_{\sigma_0}(\tau_0))\]
La aplicación $T_X$ induce un homomorfismo de $S_0(X)$ en $S_1(X\times I)$ de
forma única.

Sea $X$ un espacio topológico arbitrario y $n > 0$. Supongamos construida para
todo $i < n$ una aplicación $T_X: S_i(X) \to S_{i+1}(X\times I)$ que verifique las
siguientes condiciones:
\begin{enumerate}
\item $\p\circ T_X+T_X\circ\p=\alpha_\#-\beta_\#$ ($\star$) \label{AlfaBetaHomo},
\item $T_X\circ h_\#=(h_\#\times 1_I)\circ T_X$ para toda aplicación continua
$h\colon X \to Y$.
\end{enumerate}

Dado un $d \in S_n(\sigma_n)$, $\p d \in S_{n-1}(\sigma_n)$, por lo que la cadena
singular
\begin{equation}
c=\alpha_\#(d)-\beta_\#(d)-T_{\sigma_n}(\p d)\label{PasoInductivo}
\end{equation}
está bien definida por hipótesis de inducción. Si calculamos $\p c$,
\begin{align*}
\p c&=\p\alpha_\#(d)-\p\beta_\#(d)-\p T_{\sigma_n}(\p d)
\stackrel{\hyperref[AlfaBetaHomo]{(\star)}}{=}\\
	&=\p\alpha_\#(d)-\p\beta_\#(d)-\p\alpha_\#(d)+\p\beta_\#(d)+
	T_{\sigma_n}(\p^2 d)=0
\end{align*}
por lo que $c \in Z_n(\sigma_n\times I)$. Dado que $\sigma_n\times I$ es convexo,
\[H_n(\sigma_n\times I)=
	0 \iff Z_n(\sigma_n\times I)=
	B_n(\sigma_n\times I)\]
por lo que $c \in B_n(\sigma_n\times I)$ Existirá entonces un $b \in
S_{n+1}(\sigma_n\times I)$ tal que $\p b=c$. Se define $T_{\sigma_n}(d)=b$.

Al igual que hicimos en el caso $n=0$, definimos
\[T_X(\phi)=(\phi\times \id_I)_\#(T_{\sigma_n}(\tau_n))\]
Nos queda ver que $T_X$ verifica las dos condiciones descritas en la hipótesis de
inducción.

Veamos que $T_{\sigma_n}$ verifica la condición
$\hyperref[AlfaBetaHomo]{(\star)}$:
\begin{align*}
\p T_{\sigma_n}(d)+T_{\sigma_n}(\p d)&=
\p b+T_{\sigma_n}(\p d)=
c+T_{\sigma_n}(\p d)
\stackrel{\eqref{PasoInductivo}}{=}\\
&=\alpha_\#(d)-\beta_\#(d)-
T_{\sigma_n}(\p d)+T_{\sigma_n}(\p d)=\\
&=\alpha_\#(d)-\beta_\#(d)
\end{align*}
Como la elección de $d$ es arbitraria, se deduce que
$\p\circ T_{\sigma_n}+T_{\sigma_n}\circ\p=\alpha_\#-\beta_\#$.

Sea $\phi\colon \sigma_p \to X$ un $p$-símplice singular. Antes de probar la
primera condición, notar que $\phi_\#(\tau_n)=\phi\circ\tau_n=\phi$, por lo que
\begin{equation}
(T_X\circ\phi_\#)(\tau_n)=T_X(\phi)=(\phi\times 1_I)_\#T_{\sigma_n}(\tau_n)
\label{TConmutaTau}
\end{equation}
De esta forma,
\begin{align*}
\p T_X(\phi)+T_X(\p \phi)
	&=[\p\circ (\phi\times \id_I)_\#\circ T_{\sigma_n}](\tau_n)+
	(T_X\circ\p\circ\phi_\#)(\tau_n)=\\
	&=[\p\circ(\phi\times \id_I)_\#\circ T_{\sigma_n}](\tau_n)+
	(T_X\circ \phi_\#)(\p\tau_n)\stackrel{\eqref{TConmutaTau}}{=}\\[8pt]
	&=[\p\circ(\phi\times \id_I)_\#\circ T_{\sigma_n}](\tau_n)+
	[(\phi\times 1_I)_\#\circ T_{\sigma_n}](\p\tau_n)=\\[8pt]
	&=[(\phi\times \id_I)_\#\circ\p\circ T_{\sigma_n}](\tau_n)+
	[(\phi\times 1_I)_\#\circ T_{\sigma_n}](\p\tau_n)=\\
	&=[(\phi\times \id_I)_\#\circ(\p\circ T_{\sigma_n}+
	T_{\sigma_n}\circ\p)](\tau_n)
	\stackrel{\hyperref[AlfaBetaHomo]{(\star)}}{=}\\[8pt]
	&=(\phi\times \id_I)_\#(\alpha_\#-\beta_\#)(\tau_n)
\end{align*}
Observamos que los diagramas
\begin{center}
\begin{minipage}{0.4\textwidth}
\begin{diagram}
\sigma_n \arrow{d}{\phi} \arrow{r}{\alpha} &
\sigma_n\times I \arrow{d}{\phi\times \id_I}\\
X \arrow{r}{\alpha}& X\times I&
\end{diagram}
\end{minipage}
%
\begin{minipage}{0.4\textwidth}
\begin{diagram}
\sigma_n \arrow{d}{\phi} \arrow{r}{\beta} &
\sigma_n\times I \arrow{d}{\phi\times \id_I}\\
X \arrow{r}{\beta}& X\times I&
\end{diagram}
\end{minipage}
\end{center}
son conmutativos, por lo que también conmutan cuando los
convertimos en aplicaciones de cadenas. De aquí se sigue que
\[[(\phi\times \id_I)_\#\circ(\alpha_\#-\beta_\#)](\tau_n)=
[(\alpha_\#-\beta_\#)\circ\phi_\#](\tau_n)=(\alpha_\#-\beta_\#)(\phi)\]
por lo que $\p\circ T_X+T_X\circ\p=\alpha_\#-\beta_\#$.

Para la segunda condición, consideramos $h\colon X \to W$ continua y $\phi \in
S_n(X)$:
\begin{align*}
T_W[h_\#(\phi)]&=
	(T_W \circ h_\#\circ \phi_\#) (\tau_n)=
	[T_W\circ(h\circ \phi)_\#](\tau_n)\stackrel{\eqref{TConmutaTau}}{=}\\[8pt]
	&=[(h\circ\phi)\times \id_I]_\#[T_{\sigma_n}(\tau_n)]=\\[8pt]
	&=[(h\times \id_I)\circ(\phi\times \id_I)]_\#[T_{\sigma_n}(\tau_n)]=\\
	&=[(h\times \id_I)_\#\circ(\phi\times \id_I)_\#][T_{\sigma_n}(\tau_n)]
	\stackrel{\eqref{TConmutaTau}}{=}\\[8pt]
	&=(h\times 1_I)[T_X(\phi)]
\end{align*}
por lo que se cumple la segunda condición.

De esta forma, se construye una homotopía de cadenas entre $\alpha$ y $\beta$.
\end{proof}

Sea $f\colon X \to Y$ una aplicación continua. Decimos que $f$ es una
\textbf{equivalencia de homotopía} si existe una aplicación $g\colon Y \to X$,
llamada \textbf{inversa homotópica}, tal que
\begin{align*}
f\circ g \simeq 1_Y; && g \circ f \simeq 1_X
\end{align*}

\begin{corollary}\label{Contractil}
Si $f\colon X \to Y$ es una equivalencia de homotopía,
\[f_*: H_*(X) \to H_*(Y)\]
es un isomorfismo de grado 0. En particular, si $X$ es un espacio contráctil,
\[H_n(X)\cong H_n(\{\star\})\cong
\begin{cases}
\mb{Z} 	& \text{ si $n=0$}\\
0		& \text{ si $n \neq 0$}
\end{cases}\]
\end{corollary}

\section{Sucesiones exactas}
\begin{definition}
Una colección de grupos y homomorfismos
\[G_0 \xrightarrow{ f_1 } G_1 \xrightarrow{ f_2 } G_2 \xrightarrow{ f_3 }
\dots \xrightarrow{ f_n } G_n\]
es una \textbf{sucesión exacta finita} si $\im f_j=\ker f_{j+1}$ para todo $j$.
\end{definition}

Si $n=4$ y $G_0=0=G_4$, se obtiene una sucesión exacta finita de la forma
\[0 \rightarrow G_1 \xrightarrow{ f_2 } G_2
\xrightarrow{ f_3 } G_3 \rightarrow 0\]
En este caso, hablamos de \textbf{sucesión exacta corta}. Notar que, por
definición de sucesión exacta, $f_2$ es un monomorfismo (su núcleo es el grupo 0)
y $f_3$ es un epimorfismo (su imagen es todo $G_3$).

\begin{definition}
Una colección de grupos y homomorfismos
\[G_0 \xrightarrow{ f_1 } G_1 \xrightarrow{ f_2 } G_2
\xrightarrow{ f_3 } \dots \xrightarrow{ f_n } G_n \xrightarrow{f_{n+1}} \dots\]
es una \textbf{sucesión exacta larga} si $\im f_j=\ker f_{j+1}$ para todo $j$.
\end{definition}

Sean $C,D,E$ complejos de cadenas. Considérese la sucesión exacta corta
\begin{equation}
0 \to C \xrightarrow{ f } D \xrightarrow{ g } E \to 0 \label{SecExactafg}
\end{equation}
siendo $f$ y $g$ aplicaciones de cadenas de grado 0. Por ser aplicaciones de
cadenas, dado un $p \geq 0$, podemos construir el diagrama
\[H_p(C) \xrightarrow{ f_* } H_p(D) \xrightarrow{ g_* } H_p(E)\]
Esta secuencia no es extacta, ya que $f_*$ (resp. $g_*$) podría no ser inyectiva
(resp. sobreyectiva). Lo que sí se cumple es que $\im f_*=\ker g_*$.

Para ver esta identidad, sea $c \in Z_p(C)$. Usando que \eqref{SecExactafg} es
una secuencia exacta,
\[(g\circ f)_*([c])=[(g\circ f)(c)]=0\]
por lo que $\im f_* \leq \ker g_*$.

Análogamente, sea $[z] \in \ker g_* \leq H_p(D)$. Sabemos que $[g(z)]=0$, por
lo que $g(z)=\p d$ para algún $d \in D_{p+1}$. Como $g\colon D_{p+1} \to
E_{p+1}$ es sobreyectivo, podemos hallar un $y \in D_{p+1}$ tal que $d=g(y)$.
Dado que $g$ es aplicación de cadenas,
\[g(z)=\p d=\p g(y)=g(\p y) \implies z-\p y \in \ker g\]

Usando $\ker g=\im f$, existirá un $x \in C_p$ tal que $z-\p y=f(x)$. Como
$[z] \in H_p(D)$, $z \in Z_p(D)$, luego
\[f(\p x)=\p f(x)=\p z-\p^2 y=0 \implies \p x \in \ker f\]
Pero $f$ es un monomorfismo, luego $\p x =0$ y $x \in Z_p(C)$. Esto nos permite
tomar clases módulo $B_p(C)$: usando $f(x)=z-\p y$,
\[f_*([x])=[z-\p y]=[z]\]

Por tanto, $[z] \in \im f_*$, luego $\ker g_* \leq \im f_*$.

Para poder generar una sucesión exacta larga, vamos a introducir un homomorfismo
$\Delta\colon H_*(E) \to H_*(C)$ (llamado \textbf{homomorfismo de conexión}) que
nos permita conectar $H_n(D)$ con $H_{n-1}(C)$. El resultado será una secuencia
exacta
\[\dots \xrightarrow{\Delta} H_n(C) \xrightarrow{ f_* } H_n(D) \xrightarrow{ g_* } H_n(E)
\xrightarrow{ \Delta } H_{n-1}(C)\xrightarrow{f_*}\dots\]

Sea $z \in Z_n(E)$. Como $g\colon D \to E$ es un epimorfismo de grado 0, existe
un $d \in D_n$ tal que $z=g(d)$. Como $z\in \ker \p$ y $g$ es aplicación de
cadenas,
\[g(\p d)=0 \implies \p d \in \ker g=\im f\]
A su vez, $f\colon C \to D$ es un homomorfismo de grado 0. Existirá un $c \in
C_{n-1}$ tal que $\p d=f(c)$. Pero $f$ es aplicación de cadenas inyectiva, por
lo que
\[0=\p d=\p f(c)=f(\p c) \implies c \in \ker f=0 \implies c \in Z_{n-1}(C)\]
Esto define una correspondencia $\delta\colon z \mapsto c$ que depende
implícitamente de $d$.

\begin{proposition}
La aplicación $\Delta\colon H_*(Z) \longrightarrow H_*(C)$ dada por
$\Delta([z])=[\delta c]$ está bien definida y es un homomorfismo.
\end{proposition}

\begin{proof}
Sean $z,z' \in Z_n(E)$ con $z-z' \in B_n(E)$. Existe un $e \in E_{n+1}$ tal que
$\p e=z-z'$.

Sean $d,d' \in D_n$ tales que $g(d)=z$ y $g(d')=z'$, y sean $c,c' \in C_n$ tales
que $\p d=f(c)$ y $\p d'=f(c')$. Dado que $g$ es un epimorfismo, existe un
$a \in D_{n+1}$ tal que $g(a)=e$. Como $g$ es aplicación de cadenas,
\[g(\p a)=\p e=z-z'=g(d-d')\]
de donde se sigue que $d-d'-\p a\in \ker g=\im f$. Podemos elegir entonces un
$b \in C_n$ tal que $f(b)=d-d'-\p a$. Usando que $f$ es aplicación de cadenas,
\[f(\p b)=\p(d-d'-\p a)=\p (d- d')=f(c-c')\]
por lo que $\p b-c-c'\in\ker f$.

Usando que $f$ es inyectiva, concluimos que $c-c'=\p b \in B_{n-1}(C)$.
\end{proof}

\begin{theorem}\labthm{SucExacHomo}
Sean $C, D, E$ complejos de cadenas y
\[0 \to C \xrightarrow{ f } D \xrightarrow{ g } E \to 0\]
una sucesión exacta corta. Si $\Delta\colon H_*(E) \to H_*(C)$ es un homomorfismo
de conexión,
\[H_n(C) \xrightarrow{ f_* } H_n(D) \xrightarrow{ g_* } H_n(E)
\xrightarrow{ \Delta } H_{n-1}(C)\] es una sucesión exacta larga.
\end{theorem}

\begin{proof}
La demostración consiste en dos pasos: probar que $\im g_*=\ker \Delta$ y que
$\im \Delta = \ker f_*$. Empecemos por la primera igualdad: dado un $[z] \in \im
g_*$, existe un $d \in Z_n(D)$ tal que $[g(d)]=[z]$. Dado que $g$ es aplicación
de cadenas y $z$ es un $n$-ciclo,
\[0=\p z=g(\p d) \implies \p d \in \ker g_*=\im f_*\]
Existirá un $c \in C_{n-1}$ de forma que $\p d=f(c)$, por lo que $\Delta([z])=
[c]$. Por otro lado,
\[d \in Z_n(D) \implies \p d=0 \implies f(c)=0\]
pero $f$ es un monomorfismo, así que $c=0$. Esto nos lleva a que $z \in
\ker \Delta$.

Sea $[z] \in \ker \Delta$. Existe un $d \in D_n$ de forma que $g(d)=z$ y $f(c)=
\p d$. En particular, podemos tomar $c=0$, ya que la clase de $c$ será la del 0.
Esto implica que
\[\p d=0 \implies d \in Z_n(D) \iff [d] \in H_n(D)\]
De aquí se sigue que $g_*([d])=[g(d)]=[z]$.

Sea $[c] \in \im \Delta$. Existe un $z \in Z_n(E)$ tal que $[c]=\Delta([z])$, luego
existirá un $d \in D_n$ de forma que $f(c)=d$ y $g(d)=z$. Ahora bien,
	\[f(c)= \p d \implies f(c) \in B_n(D) \iff [f(c)]=0\]
Pero $[f(c)]=f_*([c])$, luego $[c] \in \ker f_*$.

Sea $[c] \in \ker f_*$.
Sabemos que $f(c) \in B_n(D)$, luego existirá un $d \in D_n$ tal que $f(c)=\p d$. Sea $z=g(d) \in E_n$.
Como $\ker g=\im f$, $\p z=g(\p d)=(g\circ f)(c)=0$.
	De aquí se tiene que $z \in Z_n(E)$ y
\[\Delta([z])=[c] \implies [c] \in \im \Delta\]
\end{proof}

%\input{chapters/mater_vietoris_sequence}



\pagelayout{wide} % No margins
\addpart{Homología relativa y celular}
\pagelayout{margin} % Restore margins
%\setchapterpreamble[u]{\margintoc}

\chapter{Grupos de homología relativa}
Como vinos en el capítulo anterior, el grupo de homología de orden $1$ de la
rosa tiene un generador por cada pétalo. Esto parece decirnos que la acción
de \emph{añadir un pétalo} (\textbf{adjunción}) provoca la aparición de
nuevos generadores. No obstante, diferentes adjunciones provocan diferentes
alteraciones:

\begin{itemize}
\item Si pegamos sólo uno de los extremos del segmento a $B_p$ y dejamos el
otro libre, $B_p$ es un retracto por deformación fuerte de la figura
resultante, de forma que no aparecen nuevos generadores.
\item Si lo pegamos por ambos lados, aparece un pétalo nuevo y un nuevo
generador.
\end{itemize}

Dado un espacio topológico $X$ y un subespacio $A \subseteq X$, el grupo de
homología relativa estudia cómo $A$ está pegado con su complementario,
$X\backslash A$. Diremos que dos cadenas son \textit{iguales módulo $A$} si
su diferencia es una cadena en $A$. En particular, tendremos que una cadena
será un \textit{ciclo módulo $A$} si su borde está contenido en $A$.

\section{Complejo de cadenas cociente}
\begin{definition}
Sea $(C,\partial)$ un complejo de cadenas y $(D,\partial')$ un subcomplejo de
cadenas, con $D$ subgrupo normal de $C$. Se define el \textbf{complejo
cociente} como el grupo graduado $C/D$ dotado del operador borde
\begin{diagram}
\frac{C_p}{D_p} \arrow[r]& \frac{C_{p-1}}{D_{p-1}}\\[-8mm]
\overline c \arrow[r, maps to] & \overline{\p c}
\end{diagram}
\end{definition}

Si $\pi\colon C \to C/D$ es el epimorfismo canónico e $i\colon D
\hookrightarrow C$ es la inclusión, la sucesión
\[0 \longrightarrow D \xrightarrow{i} C \xrightarrow{\pi} \frac{C}{D}
\longrightarrow 0\]
es exacta. Según el \refthm{SucExacHomo}, si podemos hallar un
homomorfismo de conexión $\Delta\colon H_*(C/D) \to H_*(C)$, la sucesión
\begin{equation}
\label{SECociente}
H_*(D) \xrightarrow{i_*} H_*(C) \xrightarrow{\pi_*}
H_*(C/D) \xrightarrow{\Delta} H_*(D)
\end{equation}
será exacta.

Para construir el homomorfismo de conexión, sea $\overline c \in Z_n(C/D)$.
Tenemos que
\[\overline{\p c}=0 \implies \partial c \in D_{n-1}\]
Como $\p^2c=0$, $\p c \in Z_{n-1}(D)$, luego $[\partial c] \in H_{n-1}(D)$.
Definimos entonces el homomorfismo $\Delta_n\colon H_n(C/D) \to H_{n-1}(D)$
como
\[\Delta_n(\overline c)=[\p c]\]
de forma que $\Delta=\{\Delta_n: n \in \mb{Z}\}$ es un homomorfismo de
conexión.

\begin{lemma}[Teorema de isomorfia para grupos graduados]\lablemma{TTIGG}
Sea $C$ un complejo de cadenas y $E,D$ subcomplejos normales de $C$. Si $E$
es un subcomplejo de $D$,
\[\frac{C}{D} \cong \frac{C/E}{D/E}\]
\end{lemma}

\marginnote[-2.2cm]{
\begin{kaobox}[frametitle=Tercer teorema de isomorfía]
Sea $C$ un grupo y $D,E$ subgrupos normales de $C$. Si $E \leq D$, $E$ es
un subgrupo normal de $D$ y
\[\frac{C}{D}\cong\frac{C/E}{D/E}\]
\end{kaobox}
}

\begin{proof}
Sea $p \in \mb{Z}$. Por definición de subcomplejo y subcomplejo normal,
$D_p,E_p \trianglelefteq C_p$ y $E_p \leq D_p$. Por el tercer teorema de
isomorfia, existe un isomorfismo
\[f_p\colon \frac{C_p}{D_p} \longrightarrow \frac{C_p/E_p}{D_p/E_p}\]

Si $f$ es un homomorfismo graduado,
\[f\colon \frac{C}{D} \longrightarrow \frac{C/E}{D/E}\]
es un isomorfismo graduado por ser una colección de isomorfismos. Por tanto,
$C/D$ es isomorfo a $(C/E)/(D/E)$.
\end{proof}

Sea $C$ un complejo de cadenas y $E \leq D$ subcomplejos normales de $C$. Por
el \reflemma{TTIGG}, existe un isomorfismo graduado $f\colon (C/E)/(D/E)
\to C/D$. Si ahora consideramos el epimorfismo canónico
\[\pi\colon C/E\longrightarrow \frac{C/E}{D/E}\]
la aplicación
\[\Pi\colon \frac{C}{E} \xrightarrow{\pi} \frac{C/E}{D/E} \xrightarrow{f}
\frac{C}{D}\]
es un epimorfismo graduado entre $C/E$ y $C/D$. De aquí se
obtiene la sucesión exacta corta
\[0 \longrightarrow \frac{D}{E} \xrightarrow{i} \frac{C}{E}
\xrightarrow{\Pi}\frac{C}{D} \longrightarrow 0\]

Sea $p\colon D \to D/E$ el epimorfismo canónico. Este homomorfismo induce un
homomorfismo de grado 0 en homología,
\[p_*\colon H_*(D) \longrightarrow H_*(D/E)\]
Si $\Delta\colon H_*(C/D) \to H_*(D)$ es el homomorfismo graduado
\eqref{SECociente},
\begin{equation}
\label{DeltaCPrima}
\Delta'\colon H_*(C/D) \xrightarrow{\Delta} H_*(D) \xrightarrow{p_*} H_*(D/E)
\end{equation}
es un homomorfismo de conexión, por lo que el \eqref{SucExacHomo} nos dice
que
\begin{equation}
\label{SECociente2}
H_n(D/E) \xrightarrow{i_*} H_n(C/E) \xrightarrow{\Pi_*} H_n(C/D)
\xrightarrow{\Delta'} H_{n-1}(D/E)
\end{equation}
es una sucesión exacta larga.

Sea $C'$ otro complejo de cadenas, con $E'\leq D'$ subcomplejos normales de
$C'$. Considérese una aplicación de cadenas $g\colon C \to C'$ tal que
$g(D) \leq D'$ y $g(E) \leq E'$. Como acabamos de ver, se puede construir
una sucesión exacta corta
\[0 \longrightarrow \frac{D'}{E'} \longrightarrow \frac{C'}{E'}
\longrightarrow \frac{C'}{D'} \longrightarrow 0\]
que induce una sucesión exacta larga en homología similar a
\eqref{SECociente2}. Dado que $g$ es una aplicación de cadenas, podemos
conectar ambas sucesiones exactas usando $g_*$:
\begin{diagram}
H_n(D/E) \arrow{r}{i_*} \arrow{d}{g_*} &
H_n(C/E) \arrow{r}{\Pi_*} \arrow{d}{g_*} &
H_n(C/D) \arrow{r}{\Delta'} \arrow{d}{g_*} &
H_{n-1}(D/E) \arrow{d}{g_*}\\
H_n(D'/E') \arrow{r}{\tilde i_*} &
H_n(C'/E') \arrow{r}{\tilde\Pi_*} &
H_n(C'/D') \arrow{r}{\tilde\Delta'} &
H_{n-1}(D'/E')
\end{diagram}

Por cómo está construida \eqref{SECociente2}, se puede probar que cada uno
de los cuadrados del diagrama anterior es conmutativo. Se dice entonces que
$g_*$ es una \textbf{transformación natural} o que cumple la
\textbf{hipótesis de naturalidad}.

\section{Homología singular relativa}
\begin{definition}
Un \textbf{par de espacios} es un par ordenado de la forma $(X,A)$, donde
$X$ es un espacio topológico y $A \subseteq X$.
\end{definition}

Sea $(X,A)$ un par de espacios y $n > 0$. Dado un $c \in S_n(X)$, existen
$n$-símplices singulares $\phi_1,\dots,\phi_p\colon \sigma_n \to X$ y enteros
$\mu_1,\dots,\mu_p$
\[c=\sum^p_{i=1}\mu_i\phi_i\]
Si $\phi_i(\sigma_n) \subseteq A$ para cada $i=1,\dots,p$, se tiene entonces
que $c \in S_n(A)$. De aquí se sigue que $S_n(A) \leq S_n(X)$.

Por otro lado, dado un $i=0,\dots,n$,
\[\p_{(i)} \phi(\sigma_{n-1}) \subset \phi(\sigma_n) \subset A
\implies \p_{(i)}\phi \in S_{n-1}(A)\]
por lo que $S_*(A)$ es un subcomplejo de cadenas de $S_*(X)$. En el lenguaje
de la teoría de categorías, decimos que $S_*$ es un \emph{funtor covariante},
la que respeta el orden de las inclusiones.

Se define el \textbf{complejo de cadenas singulares de $X$ módulo $A$} como
el complejo cociente
\[S_*(X,A):=\frac{S_*(X)}{S_*(A)}\]
De la misma forma, se definen los grupos graduados
\begin{align*}
Z_*(X,A):=\frac{Z_*(X)}{Z_*(A)};&&
B_*(X,A):=\frac{B_*(X)}{B_*(A)};
\end{align*}

\begin{definition}
Sea $(X,A)$ un par de espacios. Se define el \textbf{grupo $n$-ésimo de
homología singular relativa de $X$ módulo $A$} como
\[H_n(X,A):=H_n(S_*(X,A))=\frac{Z_n(X,A)}{B_n(X,A)}\]
\end{definition}

\marginnote[-2.2cm]{
\begin{kaobox}[frametitle=Espacio cociente]
Si $X$ es un espacio topológico y $A \subset X$, $X/A$ es el espacio
resultante de identificar todos los puntos de $A$ en uno. La topología de un
espacio cociente se toma de forma que la proyección
\[\pi\colon X \longrightarrow X/A\]
sea continua. En particular, $U \subset X/A$ es abierto si y sólo si
$\pi^{-1}(U)$ es abierto en $X$.
\end{kaobox}
}

\begin{example}\labexample{CilindroEsfera}
\begin{enumerate}
\item Sea $X$ un cilindro con tapas, y $A$ la unión de las dos tapas.
Consideramos sobre $X$ la circunferencia $e$, paralela a las anillas del
cilindro, y un segmento $f$ que conecta las dos componentes conexas de $A$.

Por un lado, el camino $e$ es un ciclo en $X$ por ser un lazo, y también lo
es en $(X,A)$. Por otro lado, el segmento $f$ no es un ciclo en $X$, pero
sí lo es en $(X,A)$.
\item Sea $X$ la esfera, y $c,d \subset X$ dos circunferencias paralelas. Si
$d$ es la circunferencia mayor, llamaremos $A$ a la componente conexa de $X
\backslash d$ que no contiene a $c$.

Usando el \refexample{Camino1Ciclo}, $c$ forma un ciclo en $X$ por ser un
lazo, por lo que $\p c=0$. Por cómo se define el operador borde del complejo
de cadenas cociente,
\[\p\overline c=\overline{\p c}=0\]
por lo que $c$ forma un ciclo en $(X,A)$.

Sea $e$ un camino que une $c$ y $d$ en $X$, y $\tilde{A}=A\cup d$. El camino
$e$ no es un ciclo en $X$ porque no es un lazo, y tampoco es un ciclo en
$(X,\tilde{A})$. No obstante, sí es un ciclo en $(X,\tilde{A}\cup c)$.
\end{enumerate}
\end{example}

\begin{marginfigure}
\includegraphics{Cilindro.pdf}
\includegraphics{Esfera.pdf}
\caption{Esfera y cilindro del \refexample{CilindroEsfera}. \labfig{Esfera}}
\end{marginfigure}

\begin{definition}
Sea $(X,A)$ un par de espacios. Tomando $C=S_*(X)$ y $D=S_*(A)$ en la
sucesión exacta (\ref{SECociente}), se obtiene la \textbf{sucesión exacta
asociada al par $(X,A)$}:
\[H_*(A) \xrightarrow{i_*} H_*(X) \xrightarrow{\pi_*} H_*(X,A)
\xrightarrow{\Delta} H_*(A)\]
\end{definition}

\begin{proposition}
$i_*$ es un isomorfismo si y sólo si $H_n(X,A)=0$ para todo $n \geq 0$.
\end{proposition}

Una \textbf{tríada de espacios} es una terna $(X,A,B)$, donde $(X,A)$ y
$(A,B)$ son pares de espacios.

Sea $(X,A,B)$ una tríada de espacios. Tomando $D=S_*(A)$, $E=S_*(B)$ y
$C=S_*(X)$ en \eqref{SECociente2}, se obtiene la sucesión exacta larga
\[H_n(A,B) \xrightarrow{i_*} H_n(X,B) \xrightarrow{j_*} H_n(X,A)
\xrightarrow{\Delta'} H_{n-1}(A,B)\]
donde $i_*$ y $j_*$ son los homomorfismos inducidos en homología por las
inclusiones $i\colon S_*(A) \hookrightarrow S_*(X)$ y $j\colon S_*(B)
\hookrightarrow S_*(A)$ y $\Delta'$ es la aplicación \eqref{DeltaCPrima}.

Recordemos que el grupo libre generado por el vacío es el grupo trivial, de
forma que $S_*(\emptyset)=0$. Esto nos permite establecer un isomorfismo
natural entre $S_*(X)$ y $S_*(X,\emptyset)$:
\begin{funcion}
S_*(X) \arrow[r] & S_*(X,\emptyset)\\
c \arrow[r, maps to] & \overline c
\end{funcion}
De la misma forma, se tiene que $S_*(X,A,\emptyset) \cong S_*(X,A)$ y
$S_*(X)\cong S_*(X,\emptyset,\emptyset)$.

\section{Aplicaciones entre pares}
\begin{definition}
Sean $(X,A)$ e $(Y,B)$ dos pares de espacios topológicos. Una
\textbf{aplicación de pares}
\[f\colon (X,A) \longrightarrow (Y,B)\]
es una aplicación continua $f\colon X \longrightarrow Y$ tal que $f(A)
\subseteq B$.
\end{definition}

Si $\phi\colon \sigma_p \to X$ es un símplice singular tal que
$\phi(\sigma_p) \subset A$ y $f\colon (X,A) \to (Y,B)$ es una
aplicación de pares, $(f\circ\phi)(\sigma_p) \subset B$, por lo que
$f_\#(\phi)$ es un $p$-símplice singular de $B$. Dado que la elección de
$\phi$ es arbitraria y los $p$-símplices singulares conforman una base de
$S_*(A)$,
\[f_\#(S_*(A)) \subseteq S_*(B)\]
por lo que $f$ induce una aplicación de cadenas
\begin{funcion}
f_\#\colon S_*(X,A) \arrow[r] & S_*(Y,B)\\
\overline c \arrow[r, maps to] & \overline{f(c)}
\end{funcion}
que induce a su vez un homomorfismo entre los grupos de homología relativa,
\begin{funcion}
f_*\colon H_*(X,A) \arrow[r] & H_*(Y,B)\\
\left[\overline c\right] \arrow[r, maps to] & \left[\overline{f(c)}\right]
\end{funcion}

Sean $f,g\colon (X,A) \to (Y,B)$ aplicaciones entre pares de espacios.
Decimos que $f$ es \textbf{homotópica} a $g$ si existe una aplicación entre
pares de espacios $F\colon (X\times I, A\times I) \to (Y,B)$ tal que
$F(x,0)=f(x)$ y $F(x,1)=g(x)$. Observar que $F$, al ser aplicación de pares,
se supone continua y verifica que $F(A\times I) \subset B$.

\begin{theorem}
Si $f,g\colon (X,A) \to (Y,B)$ son homotópicas como aplicaciones de pares,
\[f_*=g_*\colon H_*(X,A) \longrightarrow H_*(Y,B)\]
\end{theorem}

Tomando $A=\emptyset=B$, se obtiene el teorema de invarianza homotópica.

\begin{proof}
Sean $i_0,i_1\colon (X,A) \to (X\times I,A\times I)$ las aplicaciones
\begin{align*}
i_0(x)=(x,0); && i_1(x)=(x,1)
\end{align*}
Tenemos que $f=F\circ i_0$ y $g=F\circ i_1$. Para probar que $f_*=g_*$,
basta con probar que $(i_0)_\#$ e $(i_1)_\#$ son homotópicas como
aplicaciones de cadenas. La demostración es análoga al teorema de invarianza
homotópica.
\end{proof}

\begin{example}
Sean $X=[0,1]$, $Y=S_1 \subset \mb{R}^2$, $A=\{0,1\}$ y $B=\{(1,0)\}$.
Considérense las aplicaciones continuas $f,g\colon X \to Y$ definidas como
\begin{align*}
f(x)=e^{2\pi i x}; && g(x)=(1,0)
\end{align*}
Las aplicaciones $f$ y $g$ son homotópicas vía
\[F(x,t)=f(x)(1-t)+t\]
y además $f(A)=g(A)=B$, pero no forman una homotopía de pares entre $(X,A)$
e $(Y,B)$.
\end{example}

\subsection{Teorema de escisión}
\begin{lemma}[Lema de los cinco, \cite{FiveLemma}]
Sean $A_1,B_1,\dots,A_5,B_5$ grupos abelianos. Considérese el diagrama
conmutativo
\begin{diagram}
A_1 \arrow{r} \arrow{d}{\alpha} & A_2 \arrow{r} \arrow{d}{\beta_1} &
A_3 \arrow{r} \arrow{d}{\gamma} & A_4 \arrow{r} \arrow{d}{\beta_2} &
A_5 \arrow{d}{\delta}\\
B_1 \arrow{r} & B_2 \arrow{r} & B_3 \arrow{r} & B_4 \arrow{r} & B_5
\end{diagram}
cuyas filas forman sucesiones exactas. Si $\alpha$ es un epimorfismo,
$\beta_1$, $\beta_2$ son isomorfismos y $\delta$ es un monomorfismo,
$\gamma$ es un isomorfismo.
\end{lemma}

\begin{theorem}[Teorema de escisión]
Sea $(X,A)$ un par de espacios y $U \subset A$ tal que $\overline{U}
\subset \mathring A$. La aplicación inclusión $i\colon
(X\backslash U,A\backslash U) \hookrightarrow (X,A)$ induce un isomorfismo
\[i_*\colon H_*(X\backslash U,A\backslash U) \longrightarrow H_*(X,A)\]
\end{theorem}

\begin{marginfigure}
\includegraphics{HoleHoleHole.png}
\caption[Agujero en un agujero en un agujero]{El teorema de escisión nos dice
que la homología relativa de $X$ módulo $A$ sólo concierne a la frontera de
$A$. La estructura global del espacio no es relevante. Imagen: \cite{Hole}.}
\end{marginfigure}

\begin{proof}
Sea $\mc{U}=\{X\backslash U, \mathring A\}$. Tenemos por hipótesis que
\[(X\backslash U)^o=X\backslash\overline{U} \supset X\backslash\mathring{A}\]
por lo que $(X,\mc{U})$ es un espacio de Mayer-Vietoris. Si $\mc{U}'=
\{\mathring A,A\backslash U\}$, $(A,\mc{U}')$ también describe un espacio de
Mayer-Vietoris.

Los homomorfismos
\begin{align*}
i\colon S_*^\mc{U}(X) \hookrightarrow S_*(X);&&
i'\colon S_*^{\mc{U}'}(A) \hookrightarrow S_*(A)
\end{align*}
forman isomorfismos entre los grupos de homología. Teniendo en cuenta que
$S_*^{\mc{U}'}(A) \leq S_*^\mc{U}(X)$, la aplicación inclusión
\[j\colon S_*^{\mc{U},\mc{U}'}(X,A)=\frac{S_*^\mc{U}(X)}{S_*^{\mc{U}'}(A)}
\hookrightarrow \frac{S_*(X)}{S_*(A)}=S_*(X,A)\]
forma una aplicación de cadenas que da lugar al diagrama conmutativo
\begin{diagram}
H_*(S_*^{\mc{U}'}(A)) \arrow{r} \arrow{d}{i'_*} &
H_*(S_*^\mc{U}) \arrow{r} \arrow{d}{i_*} &
H_*(S_*^{\mc{U},\mc{U}'}(X,A)) \arrow{r} \arrow{d}{j_*} &
H_*(S_*^{\mc{U}'}(A)) \arrow{d}{i_*}\\
H_*(A) \arrow{r} & H_*(X) \arrow{r} & H_*(X,A) \arrow{r} & H_*(A)
\end{diagram}

\marginnote[-2.2cm]{
\begin{kaobox}[frametitle=Un detalle importante]
Dados tres grupos abelianos $A,B,C$ tales que $B+C \leq A+C$,
\[\frac{A+C}{B+C}\cong \frac{A}{B}\]
\end{kaobox}
}

Como $i_*$ e $i'_*$ son isomorfismos, el lema de los cinco nos garantiza que
$j_*$ es un isomorfismo, por lo que $S_*(X\backslash U,A\backslash U) \cong 
S_*^{\mc{U},\mc{U}'}(X,A)$ y concluimos que
\[H_*(X,A)\cong H_*(S_*^{\mc{U},\mc{U}'}(X,A)) \cong
H_*(X\backslash U,A\backslash U)\]
\end{proof}

\begin{example}\labexample{ToroAlambre}
\begin{enumerate}
\item Sea $X$ la esfera de la \reffig{Esfera}, $A$ el área
comprendida entre las dos circunferencias y $U\subset A$ una circunferencia
paralela a $\p A$. Tenemos que $X\backslash U$ son dos casquetes y
$A\backslash U$ son dos bandas. Ahora bien, observar que
\[\frac{X\backslash U}{A\backslash U} \cong \frac{X}{A}\]
por lo que ambos espacios tienen el mismo tipo de homología.
\item Sea $X$ el toro de Clifford,
\[A=\{[(x,y)] \in X\colon 1/4 \leq x \leq 3/4\}\]
y $\star \in A$. El subespacio $X\backslash\{\star\}$ es un toro
perforado. Podemos tomar el agujero y ensancharlo hasta quedarnos con dos
\emph{hilos de alambre} cruzados (que son los generadores de $H_1$). Esos
hilos forman una figura ocho.

Por otro lado, $A\backslash\{\star\}$ es un cilindro perforado. Podemos tomar
la perforación y ensancharla hasta quedarnos con las anillas de los bordes y
un alambre que los une. Dicho alambre se puede retraer hasta que las dos
anillas sean tangentes, formando una figura ocho.
\end{enumerate}
\end{example}

\begin{marginfigure}
\includegraphics{ToroGeneradoresOrden1.pdf}
\caption{Generadores del grupo de homología de primer orden del toro.}
\end{marginfigure}

\section{Grupo de homología reducida}
Sea $X$ un espacio topológico. El camino constante
$\alpha\colon X \to \{\star\}$ da lugar a un homomorfismo en homologías
$\alpha_*\colon H_*(X) \to H_*(\{\star\})$. Se define el \textbf{grupo de
homología reducida} $\tilde H_*(X)$ como el núcleo de $\alpha_*$.

Si $X$ es convexo, sabemos por el \refthm{Convexo} que $H_n(X)=\{0\}$ para
todo $n > 0$, por lo que $\tilde H_n(X)=\{0\}$. Aún así, no podemos decir nada
del grupo de orden 0, porque $H_0(X)\cong \mb{Z}$.

La homología reducida está creada para poder describir de forma más sencilla
la homología de algunos espacios. Por ejemplo: todos los grupos de homología
reducida del espacio puntual son triviales, y el único grupo de homología
reducida no trivial de $S^n$ es el de orden $n$, como veremos más adelante.

En general, si un espacio es arcoconexo, su grupo de homología reducida de
orden cero es trivial.

\begin{proposition}
Sea $X$ un espacio topológico no vacío con una cantidad finita de
arcocomponentes. $\tilde{H}_0(X)$ es un grupo libre que verifica
\[\rk(\tilde{H}_0(X))=\rk(H_0(X))-1\]
\end{proposition}

\begin{proof}
Sea $c$ un $0$-ciclo de $X$. Si llamamos $X_1,\dots,X_n$ a las
arcocomponentes de $X$, existirán $a_1,\dots,a_n \in \mb{Z}$ tales que
\[c=\sum^n_{i=1}a_ix_i; \quad x_i \in X_i\]
Dado que $\alpha$ es una aplicación constante, podemos llamar $\alpha_0$ al
valor que toma en todos sus puntos, de forma que
\[0=\alpha_*([c])=\sum^n_{i=1}a_i[\alpha(x_i)]=[\alpha_0]\sum^n_{i=1}a_i\]

De aquí se deduce que $[c] \in \ker \alpha_*$ si y sólo si
\[a_n=-\sum^{n-1}_{i=1}a_i\]
por lo que $\{x_1,\dots,x_{n-1}\}$ forma un sistema generador libre de
$\tilde{H}_0(X)$. Por tanto,
\[\tilde{H}_0(X) \cong \mb{Z}^{n-1} \implies \rk(\tilde{H}_0(X))=
n-1=\mbox{rk }(H_0(X))-1\]
\end{proof}

Sea $f\colon X \to Y$ una aplicación continua. Al igual que $f$ induce un
homomorfismo entre grupos de homología, queremos ver que $f$ induce un
homomorfismo entre grupos de homología reducida. Para ello, sean $\alpha\colon
X \to \{\star\}$, $\beta\colon Y \to \{\star\}$ caminos constantes y $c=a_1x_1+
\dots+a_nx_n$ tal que $[c] \in \tilde H_n(X)$. Tenemos que
\[f_*([c])=\sum^n_{i=1}a_i[f(x_i)]=\sum^m_{j=1}b_j[y_j]\]
Dado que las aplicaciones continuas llevan conjuntos arcoconexos en conjuntos
arcoconexos,
\[\sum^n_{i=1}a_i=\sum^m_{j=1}b_j\]

Si $[c]$ está en $\ker \alpha_*$, la suma de todos los $a_i$ es 0. Como la
suma de los $a_i$ es la misma que la de los $b_j$, esto es tanto como decir
que $f_*([c])$ está en $\ker \beta_*$. Por tanto, $f_*(\ker \alpha_*) \leq
\ker \beta_*$, de forma que $f$ induce un homomorfismo entre los grupos de
homología reducida de $X$ e $Y$.

\begin{example}
Si $X$ es un espacio contráctil, $\tilde{H}_*(X)=0$.
\end{example}

\subsection{Una fórmula para la homología reducida}
\begin{lemma}[Lema de escisión]
Sean $A,B,C$ grupos abelianos. Considérese la sucesión exacta corta
\[0 \longrightarrow A \xrightarrow{f} B \xrightarrow{g} C \longrightarrow 0\]
Las siguientes afirmaciones son equivalentes:
\begin{enumerate}
\item $B$ es suma directa de $A$ con $C$;
\item existe un homomorfismo $h\colon B \to A$ tal que $f\circ h=\id_A$;
\item existe un homomorfismo $k\colon C \to B$ tal que $k\circ g=\id_C$.
\end{enumerate}
\end{lemma}

\begin{proposition}\labprop{ReducidoRelativo}
Dado un $p \in X$,
\[\tilde{H}_*(X)\cong H_*(X,\{p\})\]
\end{proposition}

\begin{proof}
Sea $i\colon \{p\} \hookrightarrow X$ la inclusión. La aplicación continua
\[\alpha: X \longrightarrow {p}\]
verifica que $\id_{\{p\}}=\alpha\circ i$, por lo que $\{p\}$ forma un
retracto de $X$ e $i_*$ es un monomorfismo.

Considérere la sucesión exacta de homología generada por el par de espacios
$(X,\{p\})$:
\[H_n(\{p\}) \xrightarrow{i_*} H_n(X) \xrightarrow{j_*}
H_n(X,\{p\}) \xrightarrow{\Delta'} H_{n-1}(\{p\})\]
Como $i_*$ es un monomorfismo, $\im \Delta'=\ker i_*=0$. Por el primer teorema
de isomorfia, $\im j_*=\ker \Delta'=H_n(X,p)$, por lo que $j_*$ es sobreyectiva.

Dado que $i_*$ es inyectiva y $j_*$ es sobreyectiva, podemos pasar a definir
una sucesión exacta corta en lugar de trabajar con una sucesión exacta
larga:
\[0 \longrightarrow H_n(\{p\}) \xrightarrow{i_*} H_n(X) \xrightarrow{j_*}
H_n(X,\{p\}) \longrightarrow 0\]

Sabemos que $\alpha_* \circ i_*=\id_{H_*(\{p\})}$, por lo que el lema de
escisión nos garantiza que existe un homomorfismo
\[\beta\colon H_n(X,\{p\}) \to H_n(X)\]
tal que $j_*\circ \beta=\id_{H_n(X,\{p\})}$. Esto es tanto como decir que
$\beta$ es inyectiva.

Se puede probar que $\im \beta=\ker \alpha_*=\tilde{H}_*(X)$, por lo que
hemos hallado un isomorfismo entre $H_*(X,\{p\})$ y $\tilde{H}_*(X)$.
\end{proof}

Sea $X$ un espacio topológico con $n$ componentes arcoconexas. Hasta ahora,
sabemos que
\[\tilde H_0(X) \cong \mb{Z}^{n-1}\]
pero queda pregunta preguntarnos qué ocurre con los grupos de orden
superior. Utilizando la \refprop{ReducidoRelativo}, tenemos para cada $p > 0$ y
el espacio puntual $\star$ que
\[\tilde{H}_p(X)\cong H_p(X,\star)=\frac{Z_p(X,\star)}{B_p(X,\star)}=
\frac{Z_p(X)/Z_p(\star)}{B_p(X)/B_p(\star)}\]
Teniendo en cuenta que $Z_p(\star)=B_p(\star)$, estamos en condiciones de
aplicar el tercer teorema de isomorfia:
\[\tilde{H}_p(X)\cong \frac{Z_p(X)/Z_p(\star)}{B_p(X)/B_p(\star)}=
\frac{Z_p(X)/B_p(\star)}{B_p(X)/B_p(\star)} \cong
\frac{Z_p(X)}{B_p(X)}=H_p(X)\]

De esta forma, llegamos a que el grupo de homología reducida sólo
\textit{reduce} el grupo de orden cero, cuya única función es contar el
número de componentes conexas.

\begin{corollary}\label{HomoReducida}
Sea $X$ un espacio topológico con una cantidad finita de arcocomponentes.
\[\tilde{H}_*(X)\cong \frac{H_*(X)}{H_*(\star)}\]
\end{corollary}

\begin{example}
$$\tilde{H}_*(B_p)\cong\frac{H_*(B_p)}{H_*(\star)}\cong
\begin{cases}\mb{Z}^p & \mbox{ si } n=1\\0 & \mbox{ si no}\end{cases}$$
\end{example}

\subsection{Homología reducida y homología relativa}
En lo sucesivo, diremos que un espacio topológico es de \textbf{tipo $C_2$}
si es compacto y $T_2$ al mismo tiempo.

\marginnote[-2.2cm]{
\begin{kaobox}[frametitle=Idea de la demostración]
En el lema, nos proporcionan una aplicación cociente $\pi$ que
envía $X$ en $X_A$ de forma continua, por lo que nuestro impulso inicial
sería definir
\[G=\pi \circ F \circ (\pi \times \id_I)^{-1}\]
El problema es que $\pi^{-1}$ no está bien definido, por lo que tampoco lo va
a estar $(\pi\times \id_I)^{-1}$. Lo que haremos será definir una $G$ tal que
\[G\circ \pi=(\pi\circ\id_I)\circ F\]
\end{kaobox}
}

\begin{lemma}\label{RetrCoci}
Sea $(X,A)$ un par de espacios donde $X$ es $C_2$ y $A$ es cerrado en $X$.
Considérese la aplicación cociente
\[\pi\colon X \to \frac{X}{A}\]
y sea $y=\pi(A) \in X/A$. Si $A$ es retracto por deformación fuerte de $X$,
$\{y\}$ es un retracto por deformación fuerte de $\pi(X)=X/A$.
\end{lemma}

\begin{proof}
Dado que $A$ es un retracto por deformación fuerte de $X$, existe una
homotopía $F\colon X \times I \to X$ tal que $F(x,0)=x$ para todo $x \in X$,
$F(a,t)=a$ para todo $a \in A$ y $t \in I$, y $F(X\times\{1\}) \subseteq A$.

Si $X_A=X/A$, queremos hallar una homotopía $G\colon X_A\times I \to X_A$ tal
que $G([x],0)=x$ para todo $x \in X_A$, $G(X_A\times\{1\})=\{y\}$ y $G(y,t)=y$
para todo $t \in I$.

Sea $G\colon X_A\times I \longrightarrow X_A$ la aplicación
\[G([x],t)=(\pi\circ F)(x,t)\]
Si $x \not\in A$, la clase de $x$ módulo $A$ es el singulete formado por el
propio punto $x$, por lo que $G$ está bien definido en $X_A\backslash\{y\}$.

Si $x \in A$, sabemos que $F(a,t)=a$ para todo $t \in I$, por lo que
\[(\pi\circ F)(a,t)=\pi(a) \in \pi(A)=\{y\}\]
De esta forma, $G$ está bien definida y que $G(y,t)=y$ para todo $t \in I$.

Sea $p \in X/A$. Si $x \in \pi^{-1}(p)$, sabemos que $F(x,0)=x$, por lo que
\begin{align*}
G(p,0)&=(\pi\circ F)(x,0)=\pi(x)=p\\
G(p,1)&=(\pi \circ F)(x,1)\in \pi(A)=\{y\}
\end{align*}

\marginnote[-2.2cm]{
\begin{kaobox}[frametitle=Detalles de la prueba]
\begin{itemize}
\item Una aplicación es continua si y sólo si la preimagen de todo cerrado es
cerrada.
\item Los subespacios cerrados de un compacto son compactos.
\item La imagen de un compacto por una aplicación continua es compacta.
\item Todo subespacio compacto de un espacio de Hausdorff es cerrado.
\end{itemize}
\end{kaobox}
}
Sólo nos queda ver que $G$ es continua. Para ello, sea $C$ un subconjunto
cerrado de $X/A$. Por continuidad, $D=(\pi\circ F)^{-1}(C)$ es un subconjunto
cerrado de $X\times I$. Como $X\times I$ es compacto, $D$ es compacto. Por
continuidad, $(\pi\times \id_I)(D)$ es compacto. Por cómo se define $G$,
\[G^{-1}(C)=(\pi\times 1_I)(D) \subseteq X_A \times I\]
es compacto. Dado que
$X_A\times I$ es un espacio de Hausdorff, $G^{-1}(C)$ es cerrado.

Dado que la elección de $C$ es arbitraria, se sigue que $G$ es continua.
\end{proof}

\begin{lemma}
Sea $X$ un espacio $C_2$. Dados dos cerrados disjuntos $C,D \subset X$,
existen abiertos $U,V \subset X$ tales que $U$ contiene a $C$, $V$ contiene
a $D$ y $U\cap V=\emptyset$
\end{lemma}

\begin{proof}
Dados $x \in C$ e $y \in D$, como $C$ y $D$ son disjuntos, $x \neq y$. Como
$X$ es un espacio de Hausdorff, existen $U_y$, $V_y$ abiertos tales que
\begin{align*}
x \in U_y; && y \in V_y; && U_y\cap V_y=\emptyset
\end{align*}
Como $D$ es un subespacio cerrado de un compacto, $D$ es compacto, por lo
que podemos hallar una familia finita de puntos $y_1,\dots,y_n \in D$ tales
que $\{V_{y_1},\dots, V_{y_n}\}$ forma un recubrimiento abierto de $D$.

Para cada $i \in \{1,\dots,n\}$, podemos hallar un abierto $U_i \subseteq X$
tal que $U_i\cap V_{y_i}=\emptyset$ y $x \in U_i$. Se  definen entonces
\begin{align*}
U_x=\bigcup_{i=1}^nU_i;&&  V_x=\bigcup_{i=1}^n V_{y_i}
\end{align*}
Notar que ambos conjuntos son abiertos y $U_x\cap V_x=\emptyset$.

Dado que $C$ es un subespacio cerrado de un compacto, $C$ es compacto, por
lo que podemos hallar una familia finita de puntos $x_1,\dots,x_n$ tales que
$\{U_{x_1},\dots,U_{x_m}\}$ forma un recubrimiento abierto de $C$.

Para terminar, considérense los abiertos
\begin{align*}
U=\bigcup _{i=1}^m U_{x_i};&&  V=\bigcup_{i=1}^m V_{x_i}
\end{align*}
Se tiene por construcción que $D \subseteq V$ y $U\cap V=\emptyset$.
\end{proof}

\begin{lemma}\lablemma{LemaB}
Sea $X$ un espacio $C_2$. Dado un cerrado $A$ contenido en un abierto $V$,
existe un abierto $W$ tal que
$$A \subseteq W \subseteq \overline{W} \subseteq V$$
\end{lemma}

\begin{proof}
Considérense los cerrados $A$ y $X\backslash V$. Por el lema anterior, existen
abiertos $U,W$ de $X$ tales que $A \subseteq W$, $X\backslash V \subseteq U$ y
$U\cap W=\emptyset$. Tenemos que $U\cap W=\emptyset$, por lo que
\[W \subseteq X\backslash U \subseteq V \implies
\overline{W} \subset \overline{X\backslash U}=X\backslash U \subseteq V\]
\end{proof}

\begin{theorem}[Teorema del retracto]\labthm{Retracto}
Sea $X$ un espacio $C_2$, $A$ un subespacio cerrado de $X$ y $\pi\colon (X,A)
\to (X/A,\pi(A))$ una aplicación cociente. Si $A$ es un retracto por
deformación fuerte de algún entorno cerrado de $A$ en $X$,
\[H_*(X,A) \cong H_*(X/A,\pi(A)) \cong \tilde{H}_*(X/A)\]
\end{theorem}

\begin{proof}
Considérese la sucesión exacta asociada a la tríada de espacios $(X,U,A)$:
\[H_*(U,A) \xrightarrow{i_*} H_*(X,A) \xrightarrow{j_*} H_*(X,U)
\xrightarrow{\Delta'} H_*(U,A)\]
Dado que $A$ es un retracto por deformación fuerte de $U$, $H_*(U,A)=0$. Por
exactitud, se sigue que $0=\im i_*=\ker j_*$ y $\im j_*=\ker \Delta'=
H_*(X,U)$. Por tanto,
\[H_*(X,A) \cong H_*(X,U)\]

Sabemos que $X$ es un espacio $C_2$ y que $A$ es cerrado. Como $U$ es
entorno de $A$, $A \subseteq \mathring U$. Usando el \reflemma{LemaB}, podemos
hallar un abierto $V \subseteq X$ tal que $A \subseteq V \subseteq
\overline{V} \subseteq \mathring U \subseteq U$. Aplicando el teorema de
escisión, la inclusión $i\colon (X\backslash V,U\backslash V) \hookrightarrow
(X,U)$ induce un isomorfismo $i_*$. Se sigue que
\[H_*(X,U)\cong H_*(X\backslash V,U\backslash V)\]

Considérese la aplicación $p=\pi|_{X\backslash V}$. Como $A \subseteq V$,
$\pi(x)$ sólo tiene un representante (el propio $x$), de forma que $p$ es un
homeomorfismo. Como $\pi(U\backslash V)=\pi(U)\backslash \pi(V)$, se sigue que
\[H_*(X\backslash V,U\backslash V)\cong
H_*(X_A\backslash \pi(V),\pi(U)\backslash \pi(V))\]

Aplicando el teorema de escisión al par $(X_A,\pi(U))$, se obtiene el
isomorfismo
\[H_*(X_A\backslash \pi(V),\pi(U)\backslash \pi(V)) \cong H_*(X_A,\pi(U))\]

Considérese la sucesión exacta asociada a la tríada de espacios\\
$(X_A,\pi(U),\pi(A))$. Un procedimiento similar al que hemos desarrollado al
inicio de la demostración nos lleva a concluir que
\[H_*(X_A,\pi(U))\cong H_*(X_A,\pi(A))\]
\end{proof}

\section{Homeomorfismo relativo}
\begin{definition}
Un \textbf{homeomorfismo relativo} es una aplicación continua entre pares de
espacios $f\colon (X,A) \to (Y,B)$ tal que
\[f\colon X\backslash A \longrightarrow Y\backslash B\]
es un homeomorfismo.
\end{definition}

\begin{example}
\begin{enumerate}
\item Si $\mc{N}=(0,0,1) \in S^2$ y $D^2$ denota a la bola cerrada unidad de
$\mb{R}^2$, existe un homeomorfismo
\[f\colon D^2\backslash \partial D^2 \longrightarrow S^2\backslash\{\mc{N}\}\]
De esta forma, $f$ induce un homeomorfismo relativo entre los pares 
$(S^2,\{\mc{N}\})$ y $(D^2,\partial D^2)$.
\item Si $C=S^1\times I$ es un cilindro y $\mc{S}=(0,0,-1) \in S^2$, existe un
homeomorfismo
\[g\colon S^1\times \mathring{I} \to S^2-\{\mc{N},\mc{S}\}\]
que identifica al cilindro sin anillas con $S^2$
\end{enumerate}
\end{example}

\begin{theorem}[Teorema del homeomorfismo relativo]\labthm{HomeoRelativo}
Sean $X$, $Y$ espacios $C_2$, $A \subseteq X$ y $B \subseteq Y$ cerrados y
$f\colon (X,A) \to (Y,B)$ un homeomorfismo relativo. Si $A$ es retracto por
deformación fuerte de algún entorno compacto de $A$ en $X$ y $B$ es retracto
por deformación fuerte de algún entorno compacto de $B$  en $Y$, $f_*$ es un
isomorfismo.
\end{theorem}

\begin{proof}
Sean $\pi\colon X \to X/A$ y $\pi'\colon Y \to Y/B$ aplicaciones cociente. Se
define la aplicación $f'\colon X/A \to Y/B$ como $f([x])=\pi'(f(x))$.
Es fácil ver que $f'$ está bien definida y es continua, por lo que da lugar
al siguiente diagrama conmutativo:
\begin{diagram}
X \arrow{r}{f} \arrow{d}{\pi} & Y \arrow{d}{\pi'}\\
X/A \arrow{r}{f'} & Y/B
\end{diagram}

Dado que $f$ es un homeomorfismo relativo, $f'$ es una biyección entre $X/A$
e $Y/B$. Como ambos son espacios $C_2$, $f'$ es un homeomorfismo, por lo que
induce un isomorfismo entre $H_*(X/A)$ y $H_*(Y/B)$.

Tenemos entonces el diagrama conmutativo
\begin{diagram}
H_*(X,A) \arrow{r}{f_*} \arrow{d}{\pi_*} & H_*(Y,B) \arrow{d}{\pi'_*}\\
H_*(X/A,\pi(A)) \arrow{r}{f'_*} & H_*(Y/B,\pi'(B))
\end{diagram}

Sabemos por el \refthm{Retracto} que $\pi'_*$ y $\pi_*$ son isomorfismos;
además, acabamos de probar que $f_*'$ es un isomorfismo. Como consecuencia,
$f_*$ es un isomorfismo.
\end{proof}

%\input{chapters/cw_complexes}
%\setchapterpreamble[u]{\margintoc}

\chapter{Homología de las superficies compactas}
Según el teorema de clasificación de superficies compactas, toda superficie
compacta y conexa es homeomorfa a $S^2$, a la suma conexa de $n$ toros o a la
suma conexa de $n$ planos proyectivos. Entre otras cosas, eso hace que las
componentes conexas de una superficie sean arcoconexas, dado que la
arcoconexión es una propiedad topológica.

Si llamamos $S_1,\dots,S_n$ a las componentes conexas de $S$, se tiene que
\[H_*(S)\cong \sum^n_{i=1}H_*(S_i)\]
por lo que podemos suponer a efectos de homología que $S$ es arcoconexa.

Por limitaciones de tiempo, nos restringiremos al caso de las superficies
orientables, que es más breve.

\subsection{Homología del $n$-toro}
\marginnote[-2.2cm]{
\begin{kaobox}[frametitle=Suma conexa de variedades]
Sea $X$ un espacio topológico Hausdorff y 2AN. Decimos que $X$ es una
$n$-variedad si, dado $p \in X$, existe un entorno abierto $U \subset X$ de
$p$ que sea homeomorfo a una bola abierta de $\mb{R}^n$.
Dadas dos $n$-variedades $X$ e $Y$, la suma conexa se define como
\[X\#Y=(X\backslash U)\cup_f (Y\backslash V)\]
siendo $U\subset X$, $V \subset Y$ abiertos y $f\colon \p U \to \p V$ continua.
\end{kaobox}
}
El $n$-toro (que denotaremos como $T_n$) se define como la suma conexa de $n$
toros. Podemos hacer la suma conexa de los representantes llanos, en cuyo caso
obtenemos una superficie como la que se puede ver en la \reffig{Bitoro}.

Si eliminamos el interior de $T_n$, la figura resultante será $B^1_{2n}$.  Si
$\gamma(t)=e^{2\pi i\theta}$, consideramos una aplicación $f\colon S^1\to
B_{2n}^1$ tal que
\[c_j=(f\circ\gamma)\left(\frac{j-1}{2n},\frac{j}{2n}\right)\]
es igual a cada uno de los pétalos de $B^1_{2n}$, con $j=1,\dots,2n$. Tenemos
entonces que $T_n=(B_{2n})_f$ (ver \reffig{Bitoro}). 

\marginnote[-2.2cm]{
\begin{kaobox}[frametitle=Ejercicio]
Comprueba que la cadena singular $c$ que estamos definiendo verifica que
$f([c])=0$ para $n=1,2,3$. Para $n=2$, puedes usar la \reffig{Bitoro} como
referencia; para $n=1,3$, es bueno dibujar el $n$-toro llano correspondiente.
\end{kaobox}
}

Vamos a estudiar la aplicación $f_*: \tilde{H}_1(S^1) \to H_1(B^1_{2n})$. El
grupo $\tilde{H}_1(S^1)$ tiene un único generador, $[c]$. Como representante
$c$, podemos elegir cualquier camino que dé una vuelta completa a $S^1$ (si no
da una vuelta completa, lo podemos contraer en un punto, por lo que está en la
clase del $0$). En particular, podemos elegir
\[c=\phi_1+\phi_2-(\phi_3+\phi_4)+\dots...+
	\phi_{2n-3}+\phi_{2n-2}-(\phi_{2n-1}+\phi_{2n})\]
donde $\phi_j(\sigma_1)=c_j$. Esta elección verifica que $f([c])=0$. Por tanto,
\begin{align*}
\im f_*=\{0\}; && \ker f_*=\tilde H_1(S^1)\cong \mb{Z}
\end{align*}

Por la proposición \refprop{CWHomo}, $H_p(T_n)\cong H_p(B^1_{2n})=0$ para todo
$p > 2$ y $H_1(T_n) \cong H_1(B_{2n})\cong \mb{Z}^{2n}$. Sólo nos queda
calcular el grupo de homología de orden 2:
\[0 \longrightarrow H_2(B^1_{2n}) \longrightarrow H_2(T_n) \longrightarrow
\tilde{H}_1(S^1) \longrightarrow 0\]
Dado que $H_2(B^1_{2n})=0$, se tiene por exactitud que
\[H_2(T_n) \cong \tilde{H}_1(S^1) \cong \mb{Z}\]

\begin{marginfigure}
\resizebox{\textwidth}{!}{
\begin{tikzpicture}
%Octógono
\draw[fill=green!50] (1,0) -- (2,0) -- (3,1) -- (3,2) -- (2,3) --
				(1,3) -- (0,2) -- (0,1) -- (1,0);

%La línea a través de la cual hacemos la suma conexa
\draw[dashed] (1,0) -- (2,3);

%Identificación de los lados
%c1
\draw[-stealth] (1,0) -- (1.5,0);
\draw (1.5,-.25) node {$c_1$};

%c2
\draw[-stealth] (2,0) -- (2.3,.3);
\draw[-stealth] (2.3,.3) -- (2.7,.7);
\draw (2.75,.25) node {$c_2$};

%c3
\draw[-stealth] (3,1) -- (3,1.5);
\draw (3.25,1.5) node {$c_3$};

%c4
\draw[-stealth] (3,2) -- (2.7,2.3);
\draw[-stealth] (2.7,2.3) -- (2.3,2.7);
\draw (2.75,2.75) node {$c_4$};

%c5
\draw[-stealth] (2,3) -- (1.75,3);
\draw[-stealth] (1.75,3) -- (1.5,3);
\draw[-stealth] (1.5,3) -- (1.25,3);
\draw (1.5,3.25) node {$c_5$};

%c6
\draw[-stealth] (1,3) -- (.8,2.8);
\draw[-stealth] (.8,2.8) -- (.6,2.6);
\draw[-stealth] (.6,2.6) -- (.4,2.4);
\draw[-stealth] (.4,2.4) -- (.2,2.2);
\draw (.25,2.75) node {$c_2$};

%c7
\draw[-stealth] (0,2) -- (0,1.75);
\draw[-stealth] (0,1.75) -- (0,1.5);
\draw[-stealth] (0,1.5) -- (0,1.25);
\draw (-.25,1.5) node {$c_7$};

%c8
\draw[-stealth] (0,1) -- (.2,.8);
\draw[-stealth] (.2,.8) -- (.4,.6);
\draw[-stealth] (.4,.6) -- (.6,.4);
\draw[-stealth] (.6,.4) -- (.8,.2);
\draw (.25,.25) node {$c_8$};
\end{tikzpicture}
}
\resizebox{\textwidth}{!}{
\begin{tikzpicture}
\draw[fill=green!50] (0,0) -- (0,2) -- (2,2) -- (2,0) -- (0,0);
\draw[dashed] (1,2) -- (2,1);
\draw (1,1) node {$T_1$};
\draw (2.3,1.5) node {$U$};

\draw[fill=green!50] (4,0) -- (4,2) -- (6,2) -- (6,0) -- (4,0);
\draw[dashed] (4,1) -- (5,2);
\draw (5,1) node {$T_1$};
\draw (3.7,1.5) node {$V$};

\begin{scope}
\clip (1,2) -- (5,2) -- (4,1) -- (2,1) -- (1,2);
\draw[pattern=north east lines] (1,1) -- (1,2) -- (2,2) -- (2,1) -- cycle;
\draw[pattern=north west lines] (4,1) -- (4,2) -- (6,2) -- (6,1) -- cycle;
\end{scope}

\draw[-stealth] (1.5,2.25) .. controls (2.5,2.75) and (3.5,2.75) .. (4.5,2.25);
\draw (3,2.25) node {$f$};
\end{tikzpicture}
}
\caption[Bitoro llano.]{\labfig{Bitoro} Bitoro llano. Los lados con el mismo
número de cabezas de flecha se identifican entre sí. La línea discontinua
separa los dos toros que lo conforman.}
\end{marginfigure}

De esta forma, \[H_n(T_n)\cong
\begin{cases}
\mb{Z}^{2n}	&\text{ si $n =1$}\\
\mb{Z}		&\text{ si $n=0,2$}\\
0			&\text{ si no}
\end{cases}\]

\begin{theorem}
Sea $S$ una superficie compacta y orientable con $\alpha$ componentes
conexas. Existe un $n \geq 0$ tal que
\[H_q(S)\cong \begin{cases}
\mb{Z}^{2n\alpha}	&\text{ si $q =1$}\\
\mb{Z}^\alpha		&\text{ si $q =0,2$}\\
0   				&\text{ si no}
\end{cases}\]
El valor $n$ se denomina \textbf{género} de la superficie.
\end{theorem}

\subsection{Espacio proyectivo real}
\begin{definition}
Dados $v,w \in S^n$, diremos que $v \sim w$ si $w=\pm v$. Definimos el espacio
proyectivo real $\mb{R}P^n$ como el cociente $S^n/\sim$.
\end{definition}

El espacio proyectivo real se puede entender como el espacio topológico
cuyos puntos son las rectas de $\mb{R}^{n+1}$ que pasan por cero. Una forma
de entender $\mb{R}P^n$ es como el menor espacio compacto que contiene a
$\mb{R}^n$ como subespacio.

\begin{kaobox}[frametitle=¿Qué significa que $\mb{R}^n$ sea subespacio de
$\mb{R}P^n$?]
Sea $\pi\colon S^n \to \mb{R}P^n$ la proyección canónica. Dado un punto
$[x] \in \mb{R}P^n$, $\pi^{-1}([x])=\{x,-x\}$. Decimos entonces que
\emph{$S^n$ cubre a $\mb{R}P^n$ dos veces}.

Sea $B^n$ la bola abierta unidad de $\mb{R}^n$. La aplicación $f\colon
\mb{R}^n \to B^n$ dada por
\[f(x)=\frac{x}{1+\|x\|}\]
es un homeomorfismo. Por otro lado,
\begin{diagram}
g\colon B^n \arrow[r] & S^n\\[-8mm]
x \arrow[r, maps to] & \left(x,\sqrt{1-\|x\|^2}\right)
\end{diagram}
es un homeomorfismo que envía a $B^n$ en el hemisferio norte de $S^n$ (menos
el ecuador). Si $E^n_+=g(B^n)$, $\pi|_{E^n_+}$ es un homeomorfismo (observa
que $S^n$ es compacto y $\mb{R}P^n$ es Hausdorff). Si conectamos todas estas
aplicaciones, obtenemos
\begin{diagram}
h\colon \mb{R}^n \arrow{r}{f} & B^n \arrow{r}{g} & E^n_+ \arrow{r}{\pi}
	&\mb{R}P^n
\end{diagram}
Una consecuencia de nuestro proceso es que $h$ es un homeomorfismo sobre su
imagen; sin embargo, no es sobreyectiva, por lo que su imagen es un
subespacio de $\mb{R}P^n$. Ya que $\mb{R}^n$ es homeomorfo a $h(\mb{R}^n)$,
decimos por asociación que $\mb{R}^n$ es un subespacio de $\mb{R}P^n$.
\end{kaobox}

Otra posible interpretación, más en línea con la geometría sintética, es que
$\mb{R}P^n$ es una extensión de $\mb{R}^n$ donde todo par de rectas paralelas
se cruzan en \emph{el infinito}. Esta interpretación da lugar a la geometría
proyectiva, cuyo objetivo es diseñar técnicas geométricas que no dependan
de la perspectiva. Los lectores interesados pueden ver una aplicación de
geometría proyectiva en \sidecite{Numberphile}.

Podemos identificar $S^n$ con el subespacio de $S^{n+1}$ de ecuación
$x_{n+2}=0$ (el ecuador), en cuyo caso tenemos la inclusión $i\colon
S^n \hookrightarrow S^{n+1}$. Esta aplicación induce una inclusión sobre los
cocientes,
\[j\colon \mb{R}P^n \hookrightarrow \mb{R}P^{n+1}\]
En el cuadro anterior, vimos que la aplicación $\pi\circ g$ lleva a $B^{n+1}$
en un subespacio de $\mb{R}P^{n+1}$. Dado que $D^{n+1}=\overline{B^{n+1}}$,
$\pi\circ g$ induce una aplicación sobreyectiva $G\colon D^{n+1} \to
\mb{R}P^{n+1}$, que a su vez da lugar a
\[G\cup j\colon D^{n+1}\cup \mb{R}P^n \longrightarrow \mb{R}P^{n+1}\]

Dado un $p \in \mb{R}P^{n+1}$, $G^{-1}(p)$ puede ser un único punto de
$D^{n+1}\backslash S^n$ o un par $\{x,-x\}$ en $S^n$. Usando el
\reflemma{RepAdj}, concluimos que $\mb{R}P^{n+1}$ es homeomorfo al espacio
de adjunción $\mb{R}P^n_\pi$.

Para $n=1$, $\mb{R}P^1$ es homeomorfo a $S^1$, por lo que ambos espacios
tienen el mismo tipo de homología. Para $n=2$, la proyección canónica
$\pi\colon S^1 \to \mb{R}P^1$ induce una aplicación en homología $\pi_*\colon
\tilde H_1(S^1) \to H_1(\mb{R}P^1)$.

Sean $\alpha,\beta\colon I \to S^1$ las aplicaciones
\begin{align*}
\alpha(t)=(\cos \pi t, \sin\pi t); && \beta(t)=\alpha(t+1);
\end{align*}
Podemos identificar $\alpha$ y $\beta$ con $1$-símplices singulares, en cuyo
caso $[\alpha+\beta]$ es un generador de $\tilde H_1(S^1)$. Usando propiedades
de trigonometría elemental, observamos que $\pi\circ\alpha=\pi\circ\alpha$,
por lo que
\[\pi_*([\alpha+\beta])=2[\alpha]\]
y $\im \pi_*\cong 2\mb{Z}$. Usando que $\mb{Z}\cong\mb{Z}_2\times 2\mb{Z}$,
\[2\mb{Z}\cong \frac{\mb{Z}}{\ker\pi_*}
\cong\frac{2\mb{Z}\times \mb{Z}_2}{\ker \pi_*}\]
luego $\ker\pi_*\cong \mb{Z}_2$.

Usando la \refprop{HomoCW}, vemos que $H_p(\mb{R}P^2)=0$ para $p > 2$ y
\[H_1(\mb{R}P^2) \cong \frac{H_1(S^1)}{\im \pi_*}
				\cong \frac{\mb{Z}}{2\mb{Z}}=\mb{Z}_2\]
Para $n=2$, usaremos la exactitud del diagrama
\begin{diagram}
0 \arrow[r]& H_2(S^1) \arrow[r]& H_2(\mb{R}P^2) \arrow[r]&
\ker \pi_* \arrow[r]& 0
\end{diagram}
que es equivalente a
%La respuesta es 0


%\chapter{Dos generaliazciones del toro}
\section{$W=S^2\times S^1$}
El toro de Clifford se define como $S^1\times S^1$, pero podemos reemplazar una de las dos copias de $S^1$ por $S^2$ para obtener una 3-variedad.
Como veremos, esta variedad tiene característica de Euler 0, pero no es homeomorfa a $S^3$ porque sus grupos de homología no son los mismos.

Considérese la aplicación
\[\funcio{f}{\mbR}{\mb{C}}{t}{e^{2\pi i t}}\] Se define el siguiente recubrimiento de $W$: \[U=S^2\times f([0,3/4]); \quad V=S^2 \times f([1/2,5/4])\] $S^2$ es un retracto por deformación de $U$ y de $V$, por lo que $H_*(U)\cong H_*(S^2)\cong H_*(V)$ Además, $S^2\sqcup S^2$ es un retracto por deformación fuerte de $U\cap V$, por lo que $$H_*(U\cap V)\cong H_*(S^2)^2$$

Consideremos la sucesión de Mayer-Vietoris asociada al par $\{U,V\}$: dado un $n > 3$,
\[\begin{array}{ccccccc}
H_n(U\cap V)&\xrightarrow{g_n}&H_n(U)\oplus H_n(V)	&\xrightarrow{h_n}	&H_n(W)		&\xrightarrow{\Delta_n}&H_{n-1}(U \cap V)\\
\downarrow&&\downarrow			&					&\downarrow	&					&\downarrow\\
0& \rightarrow&0			&\rightarrow			&H_n(W)		&\rightarrow			&0
\end{array}\]

Por exactitud, tenemos que $h_n$ es un isomorfismo, por lo que $H_n(W) \cong H_n(U)\oplus H_n(V)=0$. Pasemos a calcular $H_1(W)$:

\[\begin{array}{ccccccc}
H_0(U\cap V)&\xrightarrow{g_0}&H_0(U)\oplus H_0(V)	&\xrightarrow{h_0}	&H_0(W)&\longrightarrow &0\\
\downarrow&					&\downarrow			&					&\downarrow & &\downarrow\\
\mb{Z}^2& \rightarrow&\mb{Z}^2			&\rightarrow			&\mb{Z} & \rightarrow & 0		
\end{array}\]

Dado que $H_0(W)\cong \mb{Z}$, $\im h_0\cong \mb{Z}$, por lo que $\im g_0=\ker h_0\cong \mb{Z}$ y $\ker g_0\cong \mb{Z}$ en consecuencia. 

\[\begin{array}{ccccccc}
H_1(U)\oplus H_1(V)	&\xrightarrow{h_1}	&H_1(W)		&\xrightarrow{\Delta_1}&H_0(U \cap V)\\
\downarrow			&					&\downarrow	&					&\downarrow\\
0			&\rightarrow			&H_1(W)		&\rightarrow			&\mb{Z}^2
\end{array}\]

Por exactitud, se tiene que $\im \Delta_1=\ker g_0\cong \mb{Z}$ y que $\ker \Delta_1=\im h_1=0$. De aquí se sigue que $$H_1(W)\cong \mb{Z}$$

Nos falta calcular $H_2(W)$ y $H_3(W)$. Para ello, consideraremos la siguiente descomposición de $S^2\times S^1$: \[U'=\{(x,y,z) \in S^2: z \geq -1/2\}; \quad U=U'\times S^1\] \[V'=\{(x,y,z) \in S^2: z \leq 1/2\}; \quad V=V'\times S^1\] Se tiene que $S^1\times S^1=T$ es un retracto por deformación de $U\cap V$, por lo que $H_*(U\cap V) \cong H_*(T)$. Además, $S^1$ es un retracto por deformación de $U$ y $V$, por lo que \[H_*(U) \cong H_*(S^1) \cong H_*(V)\]

Considérese el siguiente tramo de la sucesión de Mayer-Vietoris asociada al par $\{U,V\}$: \[\begin{array}{ccccccc}
H_3(U)\oplus H_3(V)	&\xrightarrow{h_3}	&H_3(W)		&\xrightarrow{\Delta_3}&H_2(U \cap V)&\xrightarrow{g_3}&H_2(U)\oplus H_2(V)\\
\downarrow			&					&\downarrow	&					&\downarrow & & \downarrow\\
0			&\rightarrow			&H_3(W)		&\rightarrow			&\mb{Z} & \rightarrow & 0
\end{array}\]

Por exactitud, se tiene que $\Delta_3$ es un isomorfismo entre $H_2(U\cap V) \cong \mb{Z}$ y $H_3(W)$, por lo que \[H_3(W)\cong \mb{Z}\] Sólo queda por calcular el grupo de orden 2: la aplicación $$h_0: H_0(U) \oplus H_0(V) \longrightarrow H_0(W)$$ es un epimorfismo, por lo que su imagen es isomorfa a $\mb{Z}$ y su núcleo es también isomorfo a $\mb{Z}$ (por el primer teorema de isomorfia).

\[\begin{array}{ccccccc}
H_1(U)\oplus H_1(V)	&\xrightarrow{h_1}	&H_1(W)&\xrightarrow{\Delta_1}&H_0(U\cap V)&\xrightarrow{g_0}&H_0(U) \oplus H_0(V)\\
\downarrow&		&\downarrow		&	&\downarrow & 	&\downarrow\\
\mb{Z}^2	& \rightarrow	&\mb{Z}	&\rightarrow	&\mb{Z} & \rightarrow & \mb{Z}^2	
\end{array}\]

Sabemos que $\im g_0=\ker h_0 \cong \mb{Z}$, de forma que $\im \Delta_1=\ker g_0=0$. De aquí se sigue que $\im h_1=\ker \Delta_1\cong \mb{Z}$, por lo que $\ker h_1\cong \mb{Z}$. 

\[\begin{array}{ccccccc}
H_2(U)\oplus H_2(V)	&\xrightarrow{h_2}	&H_2(W)&\xrightarrow{\Delta_2}&H_1(U\cap V)&\xrightarrow{g_1}&H_1(U) \oplus H_1(V)\\
\downarrow&		&\downarrow		&	&\downarrow & 	&\downarrow\\
0& \rightarrow	&H_2(W)		&\rightarrow	&\mb{Z}^2 & \rightarrow & \mb{Z}^2	
\end{array}\]

Dado que $\im g_1=\ker h_1\cong \mb{Z}$, $\im \Delta_2=\ker g_1\cong \mb{Z}$. También se tiene que $\ker \Delta_2=\im h_2=0$ por exactitud, de forma que $$H_2(W)\cong \frac{H_2}{\ker \Delta_2}\cong \im \Delta_2\cong \mb{Z}$$

Concluimos que \[H_n(W)\cong\begin{cases}\mb{Z} & \mbox{ si }n < 4\\0 & \mbox{ si no}\end{cases} \implies\chi(W)=1-1+1-1=0\] Es decir, que la característica de Euler del espacio $W$ coincide con la de $S^3$, pero sus grupos de homología no son isomorfos. Dado que el grupo de homología es un invariante topológico, se tiene que $S_3$ no es homeomorfo a $W$.

\section{$T^3=S^1\times S^1 \times S^1$}
Considérese la siguiente descomposición de $T^3$, siendo $f$ la aplicación del ejemplo anterior: \[U=T\times f([0,3/4]); \quad V=T \times f([1/2,5/4])\] Se tiene que $T$ es un retracto por deformación de $U$ y de $V$, por lo que $H_*(U)\cong H_*(T) \cong H_*(V)$.
Por otro lado, $U\cap V=T\sqcup T$, por lo que \[H_*(U\cap V)\cong H_*(T_2)^2\]

Para $n > 3$, la sucesión de Mayer-Vietoris asociada al par $\{U,V\}$ da lugar a la siguiente sucesión exacta corta: \[\begin{array}{ccccccc}
H_n(U\cap V)&\xrightarrow{g_n}&H_n(U)\oplus H_n(V)	&\xrightarrow{h_n}	&H_n(T^3)		&\xrightarrow{\Delta_n}&H_{n-1}(U \cap V)\\
\downarrow&&\downarrow			&					&\downarrow	&					&\downarrow\\
0& \rightarrow&0			&\rightarrow			&H_n(T^3)		&\rightarrow			&0
\end{array}\] Por exactitud, se tiene que $H_n(T^3)\cong H_n(T)^2=0$. Sólo necesitamos calcular los casos $n=1,2,3$.

\[\begin{array}{ccccccc}
H_0(U\cap V)&\xrightarrow{g_0}&H_0(U)\oplus H_0(V)	&\xrightarrow{h_0}	&H_0(T^3)&\longrightarrow &0\\
\downarrow&					&\downarrow			&					&\downarrow & &\downarrow\\
\mb{Z}^2& \rightarrow&\mb{Z}^2			&\rightarrow			&\mb{Z} & \rightarrow & 0		
\end{array}\]

Dado que $H_0(W)\cong \mb{Z}$, $\im h_0\cong \mb{Z}$, por lo que $\im g_0=\ker h_0\cong \mb{Z}$ y $\ker g_0\cong \mb{Z}$ en consecuencia.

\[\begin{array}{ccccccc}
H_1(U)\oplus H_1(V)	&\xrightarrow{h_1}	&H_1(T^3)		&\xrightarrow{\Delta_1}&H_0(U \cap V)\\
\downarrow			&					&\downarrow		&						&\downarrow\\
\mb{Z}^4			&\rightarrow		&H_1(T^3)		&\rightarrow			&\mb{Z}^2
\end{array}\]

Se tiene por exactitud que $\im \Delta_1=\ker g_0\cong \mb{Z}$. Por el primer teorema de isomorfia, \[\frac{H_1(T^3)}{\ker \Delta_1} \cong \mb{Z}\] Necesitamos determinar el núcleo de $\Delta_1$ manualmente; no obstante, podemos calcular la imagen de $h_1$ en su lugar, que es más sencillo.
\\

Recordemos cómo se define $h_1$: \[\funcio{h_1}{H_1(T)^2}{H_1(T^3)}{([\alpha],[\beta])}{[\alpha]+[\beta]}\] Dado que $H_1(T)$ tiene dos generadores, existen 1-ciclos $a,b \in Z_1(T)$ y enteros $\lambda_1,\lambda_2,\mu_1,\mu_2$ tales que \[[\alpha]+[\beta]=\lambda_1[a]+\mu_1[b]+\lambda_2[a]+\mu_2[b]\] De aquí se sigue que \[\ker \Delta_1=\im h_1=\{\lambda[a]+\mu[b]: \lambda,\mu \in \mb{Z}\} \cong \mb{Z}^2\] por lo que $H_1(T^3)\cong \mb{Z}^3$.

\[\begin{array}{ccccccc}
H_2(U)\oplus H_2(V)	&\xrightarrow{h_2}	&H_2(T^3)&\xrightarrow{\Delta_2}&H_1(U\cap V)&\xrightarrow{g_1}&H_1(U) \oplus H_1(V)\\
\downarrow&		&\downarrow		&	&\downarrow & 	&\downarrow\\
\mb{Z}^2& \rightarrow	&H_2(T^3)		&\rightarrow	&\mb{Z}^4 & \rightarrow & \mb{Z}^4
\end{array}\]

Como $\im h_1\cong \mb{Z}^2$, se tiene que $\im g_1=\ker h_1\cong \mb{Z}^2$, por lo que $\im \Delta_2=\ker g_1\cong \mb{Z}^2$. Aplicando el primer teorema de isomorfia una vez más, \[\frac{H_2(T^3)}{\ker \Delta_2} \cong \mb{Z}^2\] Una vez más, calcularemos el grupo $\im h_2$ para poder continuar.
\\

En este caso, el dominio de $h_2$ es $H_2(T)^2$. $H_2(T)$ tiene rango 1, por lo que es un subgrupo cíclico; esto quiere decir que todos sus elementos son de la forma $\mu [a]$, siendo $a \in Z_2(T)-B_2(T)$. Por tanto, si $\alpha,\beta \in H_2(T)$, \[h_2(\alpha,\beta)=\alpha+\beta=(\mu_1+\mu_2)[a]\] para algunos $\mu_1,\mu_2$ enteros. De aquí se sigue que $\ker \Delta_2=\im h_2$ es él mismo un monógeno, por lo que es isomorfo a $\mb{Z}$. Como consecuencia, $H_2(T^3)\cong \mb{Z}^3$.
\\

El grupo de orden 3 ya no requiere manipulaciones algebraicas; puede computarse utilizando exactitud.
\[\begin{array}{ccccccc}
H_3(U)\oplus H_3(V)	&\xrightarrow{h_3}	&H_3(T^3)		&\xrightarrow{\Delta_3}&H_2(U \cap V)&\xrightarrow{g_3}&H_2(U)\oplus H_2(V)\\
\downarrow			&					&\downarrow	&					&\downarrow & & \downarrow\\
0			&\rightarrow			&H_3(T^3)		&\rightarrow			&\mb{Z}^2 & \rightarrow & \mb{Z}^2
\end{array}\] Tenemos que \[\im h_2\cong \mb{Z} \implies \im g_2\cong \ker h_2\cong \mb{Z}\implies \im \Delta_3=\ker g_2\cong \mb{Z}\] Pero $\ker \Delta_3=0$ por exactitud, por lo que se concluye que $H_3(T^3)\cong \im \Delta_3 \cong \mb{Z}$. Por tanto, \[H_n(T^3)\cong \begin{cases}\mb{Z} &\mbox{ si }n=0,3\\ \mb{Z}^3 & \mbox{ si }n=1,2\\ 0&\mbox{ si }n > 3\end{cases} \implies \chi(T^3)=1-3+3-1=0\]

Tal y como habíamos predicho, la característica de Euler de $T^3$ coincide con la de $S^3$. No obstante, al igual que en el caso de la variedad $W$, tenemos que $H_*(T^3)\neq H_*(S^3)$, por lo que no son variedades homeomorfas. Pero eso es lo que queríamos demostrar. \qed


\appendix % From here onwards, chapters are numbered with letters, as is the appendix convention

\pagelayout{wide} % No margins
\addpart{Apéndices}
\pagelayout{margin} % Restore margins

%\input{chapters/double_induction}
%\input{chapters/complements}

%----------------------------------------------------------------------------------------

\backmatter % Denotes the end of the main document content
\setchapterstyle{plain} % Output plain chapters from this point onwards

%----------------------------------------------------------------------------------------
%	BIBLIOGRAPHY
%----------------------------------------------------------------------------------------

% The bibliography needs to be compiled with biber using your LaTeX editor, or on the command line with 'biber main' from the template directory

\defbibnote{bibnote}{Here are the references in citation order.\par\bigskip} % Prepend this text to the bibliography
\printbibliography[heading=bibintoc, title=Bibliography, prenote=bibnote] % Add the bibliography heading to the ToC, set the title of the bibliography and output the bibliography note

%----------------------------------------------------------------------------------------
%	INDEX
%----------------------------------------------------------------------------------------

% The index needs to be compiled on the command line with 'makeindex main' from the template directory
\printindex % Output the index

%----------------------------------------------------------------------------------------
%	BACK COVER
%----------------------------------------------------------------------------------------

% If you have a PDF/image file that you want to use as a back cover, uncomment the following lines

%\clearpage
%\thispagestyle{empty}
%\null%
%\clearpage
%\includepdf{cover-back.pdf}

%----------------------------------------------------------------------------------------

\end{document}
