\chapter{Método de doble inducción}\label{DobleInd}
El propósito de este apéndice es explicar en detalle cómo funciona la
\emph{doble inducción} utilizada en la demostración del \refthm{RosaMixta},
por qué es un método de demostración válido (cuando se usa correctamente) y
bajo qué condiciones podemos hacer \emph{inducción múltiple}.

\begin{definition}
Sea $(S,<)$ un conjunto totalmente ordenado. Dado $s \in S$, definimos el
\textbf{sucesor} de $s$ como el menor $s' \in S$ tal que $s < s'$. El sucesor
está bien definido porque el orden es total.
\end{definition}

\marginnote[-2.2cm]{
\begin{kaobox}[frametitle=Conjunto bien ordenado]
Un conjunto totalmente ordenado $(S,<)$ está bien ordenado si, dado un
subconjunto $S' \subset S$, $S'$ tiene un elemento minimal. Por ejemplo,
$\mb{N}$ está bien ordenado, pero $\mb{Z}$ no.

Un ejemplo de subconjunto que no cumple esta propiedad es el conjunto de los
enteros negativos.
\end{kaobox}
}

Si $P(n)$ es una propiedad que depende de $n \in \mb{N}$, el método de
inducción matemática se puede representar como un diagrama de la forma
\begin{diag}
P(1) \arrow[r] & P(2) \arrow[r] & \dots \arrow[r] & P(n) \arrow[r] & \dots
\end{diag}
Este método se basa en el hecho de que $\mb{N}$ es un conjunto bien ordenado
bajo la relación $\leq$ (i.e. todo conjunto no vacío de números naturales tiene
un mínimo). En general, podemos plantear un método de inducción sobre cualquier
conjunto totalmente ordenado.

\begin{definition}
Sea $(S,<)$ un conjunto totalmente ordenado. Supongamos que, dada una propiedad
$P(s)$ que depende de $s \in S$, tenemos los siguientes resultados:
\begin{enumerate}
\item $P(\min\{s\})$ es cierto;
\item si $s'$ es el sucesor de $s$, $P(s)$ implica $P(s')$.
\end{enumerate}
Diremos que $S$ admite un método de inducción si estas dos propiedades implican
que $P(s)$ es cierto para todo $s \in S$.
\end{definition}

\begin{proposition}[\cite{BienOrd}]
Sea $S$ un conjunto totalmente ordenado. El conjunto $S$ admite un método de
inducción si y sólo si está bien ordenado.
\end{proposition}

Consideremos entonces el caso de una propiedad $Q(n,m)$ que depende de $n,m \in
\mb{N}$. En este caso, nuestro conjunto de partida es $\mb{N}^2$, así que
necesitamos una relación de orden total para poder probar $Q$ por inducción.

Dados $(n_1,m_1),(n_2,m_2) \in \mb{N}$, definimos el orden lexicográfico
$\prec$ como
\[(n_1,m_1) \prec (n_2,m_2) \iff
\begin{cases}
n_1 < n_2\\
n_1 = n_2; & m_1 < m_2
\end{cases}
\]
El orden lexicográfico se puede extender a $\mb{N}^k$, y se utiliza en algunos
campos del álgebra tales como la teoría de polinomios. Dado que el orden
lexicográfico es el orden usual de $\mb{N}$ \emph{aplicado dos veces}, es fácil
ver que induce un buen orden en $\mb{N}^2$, por lo que podemos probar $Q$ de
forma inductiva.

Para poder probar $Q$, lo que tenemos que hacer es probar primero $P(n)=Q(n,1)$
por inducción sobre $n$, y después, probar $R(m)=Q(n,m)$ por inducción sobre
$m$. El diagrama correspondiente toma la forma
\begin{diag}
Q(1,1) \arrow[r] \arrow[d, dashed] & Q(2,1) \arrow[r] \arrow[d,dashed] &
	\dots \arrow[r] \arrow[d,dashed] & Q(n,1) \arrow[r] \arrow[d,dashed] &
	 \dots \arrow[d,dashed]\\
Q(1,2) \arrow[r] \arrow[d, dashed] & Q(2,2) \arrow[r] \arrow[d,dashed] &
	\dots \arrow[r] \arrow[d,dashed] & Q(n,2) \arrow[r] \arrow[d,dashed] &
	\dots \arrow[d,dashed]\\
\vdots \arrow[r] \arrow[d, dashed] & \vdots \arrow[r] \arrow[d,dashed] &
	\ddots \arrow[r] \arrow[d,dashed] & \vdots \arrow[r] \arrow[d,dashed] &
	\vdots \arrow[d,dashed]\\
Q(1,m) \arrow[r] \arrow[d, dashed] & Q(2,m) \arrow[r] \arrow[d,dashed] &
	\dots \arrow[r] \arrow[d,dashed] & Q(n,m) \arrow[r] \arrow[d,dashed] &
	\dots \arrow[d,dashed]\\
\vdots \arrow[r] & \vdots \arrow[r] & \vdots \arrow[r] & \vdots \arrow[r] &
	\dots
\end{diag}
donde las líneas sólidas representan $n_1 < n_2$, y las líneas discontinuas,
$m_1 < m_2$.