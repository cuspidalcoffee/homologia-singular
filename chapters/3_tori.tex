\chapter{Dos generaliazciones del toro}
\section{$W=S^2\times S^1$}
El toro de Clifford se define como $S^1\times S^1$, pero podemos reemplazar una de las dos copias de $S^1$ por $S^2$ para obtener una 3-variedad.
Como veremos, esta variedad tiene característica de Euler 0, pero no es homeomorfa a $S^3$ porque sus grupos de homología no son los mismos.

Considérese la aplicación
\[\funcio{f}{\mbR}{\mb{C}}{t}{e^{2\pi i t}}\] Se define el siguiente recubrimiento de $W$: \[U=S^2\times f([0,3/4]); \quad V=S^2 \times f([1/2,5/4])\] $S^2$ es un retracto por deformación de $U$ y de $V$, por lo que $H_*(U)\cong H_*(S^2)\cong H_*(V)$ Además, $S^2\sqcup S^2$ es un retracto por deformación fuerte de $U\cap V$, por lo que $$H_*(U\cap V)\cong H_*(S^2)^2$$

Consideremos la sucesión de Mayer-Vietoris asociada al par $\{U,V\}$: dado un $n > 3$,
\[\begin{array}{ccccccc}
H_n(U\cap V)&\xrightarrow{g_n}&H_n(U)\oplus H_n(V)	&\xrightarrow{h_n}	&H_n(W)		&\xrightarrow{\Delta_n}&H_{n-1}(U \cap V)\\
\downarrow&&\downarrow			&					&\downarrow	&					&\downarrow\\
0& \rightarrow&0			&\rightarrow			&H_n(W)		&\rightarrow			&0
\end{array}\]

Por exactitud, tenemos que $h_n$ es un isomorfismo, por lo que $H_n(W) \cong H_n(U)\oplus H_n(V)=0$. Pasemos a calcular $H_1(W)$:

\[\begin{array}{ccccccc}
H_0(U\cap V)&\xrightarrow{g_0}&H_0(U)\oplus H_0(V)	&\xrightarrow{h_0}	&H_0(W)&\longrightarrow &0\\
\downarrow&					&\downarrow			&					&\downarrow & &\downarrow\\
\mb{Z}^2& \rightarrow&\mb{Z}^2			&\rightarrow			&\mb{Z} & \rightarrow & 0		
\end{array}\]

Dado que $H_0(W)\cong \mb{Z}$, $\im h_0\cong \mb{Z}$, por lo que $\im g_0=\ker h_0\cong \mb{Z}$ y $\ker g_0\cong \mb{Z}$ en consecuencia. 

\[\begin{array}{ccccccc}
H_1(U)\oplus H_1(V)	&\xrightarrow{h_1}	&H_1(W)		&\xrightarrow{\Delta_1}&H_0(U \cap V)\\
\downarrow			&					&\downarrow	&					&\downarrow\\
0			&\rightarrow			&H_1(W)		&\rightarrow			&\mb{Z}^2
\end{array}\]

Por exactitud, se tiene que $\im \Delta_1=\ker g_0\cong \mb{Z}$ y que $\ker \Delta_1=\im h_1=0$. De aquí se sigue que $$H_1(W)\cong \mb{Z}$$

Nos falta calcular $H_2(W)$ y $H_3(W)$. Para ello, consideraremos la siguiente descomposición de $S^2\times S^1$: \[U'=\{(x,y,z) \in S^2: z \geq -1/2\}; \quad U=U'\times S^1\] \[V'=\{(x,y,z) \in S^2: z \leq 1/2\}; \quad V=V'\times S^1\] Se tiene que $S^1\times S^1=T$ es un retracto por deformación de $U\cap V$, por lo que $H_*(U\cap V) \cong H_*(T)$. Además, $S^1$ es un retracto por deformación de $U$ y $V$, por lo que \[H_*(U) \cong H_*(S^1) \cong H_*(V)\]

Considérese el siguiente tramo de la sucesión de Mayer-Vietoris asociada al par $\{U,V\}$: \[\begin{array}{ccccccc}
H_3(U)\oplus H_3(V)	&\xrightarrow{h_3}	&H_3(W)		&\xrightarrow{\Delta_3}&H_2(U \cap V)&\xrightarrow{g_3}&H_2(U)\oplus H_2(V)\\
\downarrow			&					&\downarrow	&					&\downarrow & & \downarrow\\
0			&\rightarrow			&H_3(W)		&\rightarrow			&\mb{Z} & \rightarrow & 0
\end{array}\]

Por exactitud, se tiene que $\Delta_3$ es un isomorfismo entre $H_2(U\cap V) \cong \mb{Z}$ y $H_3(W)$, por lo que \[H_3(W)\cong \mb{Z}\] Sólo queda por calcular el grupo de orden 2: la aplicación $$h_0: H_0(U) \oplus H_0(V) \longrightarrow H_0(W)$$ es un epimorfismo, por lo que su imagen es isomorfa a $\mb{Z}$ y su núcleo es también isomorfo a $\mb{Z}$ (por el primer teorema de isomorfia).

\[\begin{array}{ccccccc}
H_1(U)\oplus H_1(V)	&\xrightarrow{h_1}	&H_1(W)&\xrightarrow{\Delta_1}&H_0(U\cap V)&\xrightarrow{g_0}&H_0(U) \oplus H_0(V)\\
\downarrow&		&\downarrow		&	&\downarrow & 	&\downarrow\\
\mb{Z}^2	& \rightarrow	&\mb{Z}	&\rightarrow	&\mb{Z} & \rightarrow & \mb{Z}^2	
\end{array}\]

Sabemos que $\im g_0=\ker h_0 \cong \mb{Z}$, de forma que $\im \Delta_1=\ker g_0=0$. De aquí se sigue que $\im h_1=\ker \Delta_1\cong \mb{Z}$, por lo que $\ker h_1\cong \mb{Z}$. 

\[\begin{array}{ccccccc}
H_2(U)\oplus H_2(V)	&\xrightarrow{h_2}	&H_2(W)&\xrightarrow{\Delta_2}&H_1(U\cap V)&\xrightarrow{g_1}&H_1(U) \oplus H_1(V)\\
\downarrow&		&\downarrow		&	&\downarrow & 	&\downarrow\\
0& \rightarrow	&H_2(W)		&\rightarrow	&\mb{Z}^2 & \rightarrow & \mb{Z}^2	
\end{array}\]

Dado que $\im g_1=\ker h_1\cong \mb{Z}$, $\im \Delta_2=\ker g_1\cong \mb{Z}$. También se tiene que $\ker \Delta_2=\im h_2=0$ por exactitud, de forma que $$H_2(W)\cong \frac{H_2}{\ker \Delta_2}\cong \im \Delta_2\cong \mb{Z}$$

Concluimos que \[H_n(W)\cong\begin{cases}\mb{Z} & \mbox{ si }n < 4\\0 & \mbox{ si no}\end{cases} \implies\chi(W)=1-1+1-1=0\] Es decir, que la característica de Euler del espacio $W$ coincide con la de $S^3$, pero sus grupos de homología no son isomorfos. Dado que el grupo de homología es un invariante topológico, se tiene que $S_3$ no es homeomorfo a $W$.

\section{$T^3=S^1\times S^1 \times S^1$}
Considérese la siguiente descomposición de $T^3$, siendo $f$ la aplicación del ejemplo anterior: \[U=T\times f([0,3/4]); \quad V=T \times f([1/2,5/4])\] Se tiene que $T$ es un retracto por deformación de $U$ y de $V$, por lo que $H_*(U)\cong H_*(T) \cong H_*(V)$.
Por otro lado, $U\cap V=T\sqcup T$, por lo que \[H_*(U\cap V)\cong H_*(T_2)^2\]

Para $n > 3$, la sucesión de Mayer-Vietoris asociada al par $\{U,V\}$ da lugar a la siguiente sucesión exacta corta: \[\begin{array}{ccccccc}
H_n(U\cap V)&\xrightarrow{g_n}&H_n(U)\oplus H_n(V)	&\xrightarrow{h_n}	&H_n(T^3)		&\xrightarrow{\Delta_n}&H_{n-1}(U \cap V)\\
\downarrow&&\downarrow			&					&\downarrow	&					&\downarrow\\
0& \rightarrow&0			&\rightarrow			&H_n(T^3)		&\rightarrow			&0
\end{array}\] Por exactitud, se tiene que $H_n(T^3)\cong H_n(T)^2=0$. Sólo necesitamos calcular los casos $n=1,2,3$.

\[\begin{array}{ccccccc}
H_0(U\cap V)&\xrightarrow{g_0}&H_0(U)\oplus H_0(V)	&\xrightarrow{h_0}	&H_0(T^3)&\longrightarrow &0\\
\downarrow&					&\downarrow			&					&\downarrow & &\downarrow\\
\mb{Z}^2& \rightarrow&\mb{Z}^2			&\rightarrow			&\mb{Z} & \rightarrow & 0		
\end{array}\]

Dado que $H_0(W)\cong \mb{Z}$, $\im h_0\cong \mb{Z}$, por lo que $\im g_0=\ker h_0\cong \mb{Z}$ y $\ker g_0\cong \mb{Z}$ en consecuencia.

\[\begin{array}{ccccccc}
H_1(U)\oplus H_1(V)	&\xrightarrow{h_1}	&H_1(T^3)		&\xrightarrow{\Delta_1}&H_0(U \cap V)\\
\downarrow			&					&\downarrow		&						&\downarrow\\
\mb{Z}^4			&\rightarrow		&H_1(T^3)		&\rightarrow			&\mb{Z}^2
\end{array}\]

Se tiene por exactitud que $\im \Delta_1=\ker g_0\cong \mb{Z}$. Por el primer teorema de isomorfia, \[\frac{H_1(T^3)}{\ker \Delta_1} \cong \mb{Z}\] Necesitamos determinar el núcleo de $\Delta_1$ manualmente; no obstante, podemos calcular la imagen de $h_1$ en su lugar, que es más sencillo.
\\

Recordemos cómo se define $h_1$: \[\funcio{h_1}{H_1(T)^2}{H_1(T^3)}{([\alpha],[\beta])}{[\alpha]+[\beta]}\] Dado que $H_1(T)$ tiene dos generadores, existen 1-ciclos $a,b \in Z_1(T)$ y enteros $\lambda_1,\lambda_2,\mu_1,\mu_2$ tales que \[[\alpha]+[\beta]=\lambda_1[a]+\mu_1[b]+\lambda_2[a]+\mu_2[b]\] De aquí se sigue que \[\ker \Delta_1=\im h_1=\{\lambda[a]+\mu[b]: \lambda,\mu \in \mb{Z}\} \cong \mb{Z}^2\] por lo que $H_1(T^3)\cong \mb{Z}^3$.

\[\begin{array}{ccccccc}
H_2(U)\oplus H_2(V)	&\xrightarrow{h_2}	&H_2(T^3)&\xrightarrow{\Delta_2}&H_1(U\cap V)&\xrightarrow{g_1}&H_1(U) \oplus H_1(V)\\
\downarrow&		&\downarrow		&	&\downarrow & 	&\downarrow\\
\mb{Z}^2& \rightarrow	&H_2(T^3)		&\rightarrow	&\mb{Z}^4 & \rightarrow & \mb{Z}^4
\end{array}\]

Como $\im h_1\cong \mb{Z}^2$, se tiene que $\im g_1=\ker h_1\cong \mb{Z}^2$, por lo que $\im \Delta_2=\ker g_1\cong \mb{Z}^2$. Aplicando el primer teorema de isomorfia una vez más, \[\frac{H_2(T^3)}{\ker \Delta_2} \cong \mb{Z}^2\] Una vez más, calcularemos el grupo $\im h_2$ para poder continuar.
\\

En este caso, el dominio de $h_2$ es $H_2(T)^2$. $H_2(T)$ tiene rango 1, por lo que es un subgrupo cíclico; esto quiere decir que todos sus elementos son de la forma $\mu [a]$, siendo $a \in Z_2(T)-B_2(T)$. Por tanto, si $\alpha,\beta \in H_2(T)$, \[h_2(\alpha,\beta)=\alpha+\beta=(\mu_1+\mu_2)[a]\] para algunos $\mu_1,\mu_2$ enteros. De aquí se sigue que $\ker \Delta_2=\im h_2$ es él mismo un monógeno, por lo que es isomorfo a $\mb{Z}$. Como consecuencia, $H_2(T^3)\cong \mb{Z}^3$.
\\

El grupo de orden 3 ya no requiere manipulaciones algebraicas; puede computarse utilizando exactitud.
\[\begin{array}{ccccccc}
H_3(U)\oplus H_3(V)	&\xrightarrow{h_3}	&H_3(T^3)		&\xrightarrow{\Delta_3}&H_2(U \cap V)&\xrightarrow{g_3}&H_2(U)\oplus H_2(V)\\
\downarrow			&					&\downarrow	&					&\downarrow & & \downarrow\\
0			&\rightarrow			&H_3(T^3)		&\rightarrow			&\mb{Z}^2 & \rightarrow & \mb{Z}^2
\end{array}\] Tenemos que \[\im h_2\cong \mb{Z} \implies \im g_2\cong \ker h_2\cong \mb{Z}\implies \im \Delta_3=\ker g_2\cong \mb{Z}\] Pero $\ker \Delta_3=0$ por exactitud, por lo que se concluye que $H_3(T^3)\cong \im \Delta_3 \cong \mb{Z}$. Por tanto, \[H_n(T^3)\cong \begin{cases}\mb{Z} &\mbox{ si }n=0,3\\ \mb{Z}^3 & \mbox{ si }n=1,2\\ 0&\mbox{ si }n > 3\end{cases} \implies \chi(T^3)=1-3+3-1=0\]

Tal y como habíamos predicho, la característica de Euler de $T^3$ coincide con la de $S^3$. No obstante, al igual que en el caso de la variedad $W$, tenemos que $H_*(T^3)\neq H_*(S^3)$, por lo que no son variedades homeomorfas. Pero eso es lo que queríamos demostrar. \qed
