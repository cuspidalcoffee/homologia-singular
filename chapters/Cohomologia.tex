\setchapterpreamble[u]{\margintoc}

\chapter{Cohomología singular}

\section{Grupos de cohomología}
\begin{definition}
Un complejo de cocadenas es un par $(C,d)$ donde $C=\{C^j: j \in \mb{Z}\}$ es
un grupo abeliano graduado y $d\colon C \to C$ es un endomorfismo graduado
de grado $1$ tal que $d^{p+1}\circ d^p=0$. Es decir, tenemos una sucesión
exacta
\[0 \longrightarrow C^0 \xrightarrow{d^0} C^1 \xrightarrow{d^1} C^2
\xrightarrow{d^2}\dots\]
donde las flechas van hacia arriba, en lugar de ir hacia abajo.
\end{definition}

En este capítulo, veremos que muchos términos de homología singular se
repiten, añadiendo el prefijo \emph{co-}. Ésto se debe a que la teoría de
cohomología complementa a teoría de homología, dado que estamos utilizando
los mismos principios, pero aquí le damos la vuelta al sentido de las flechas
en las sucesiones exactas.

Al igual que hay diferentes teorías de homología (simplicial, singular,
celular...), existen diferentes teorías de cohomología que están diseñadas
para diferentes contextos. En este capítulo, cubriremos la teoría de
cohomología singular, que es la más general y no requiere ningún conocimiento
preeliminar que no hayamos visto ya.

Los elementos de $C^n$ se denominan \textbf{cocadenas} de orden $n$, y los
homomorfismos $d^n$ reciben el nombre de \textbf{operadores coborde}.

\marginnote[-2.2cm]{
\begin{kaobox}[frametitle=Coborde y cobordismo]
El \emph{coborde} que definimos en este capítulo no está relacionado con
los \emph{cobordismos}. Sin entrar en detalles, se podría decir que dos
variedades $X$ e $Y$ de dimensión $n$ son cobordantes si existe una
homotopía
\[F\colon X \simeq Y\]
cuyo grafo es una variedad con bordes de dimensión $n+1$.
\end{kaobox}
}

Definimos el grupo de $n$-cociclos como $Z^n(C)=\ker d^n$, y el grupo de $n$-
cobordes como $B^n(C)=\im d^n$. Dado que $d^{n+1}\circ d^n=0$, $B^n(C)\leq
Z^n(C)$, por lo que definimos el \textbf{grupo $n$-ésimo de cohomología} como
\[H^n(C)=\frac{Z^n(C)}{B^n(C)}\]
Notar que, para diferenciar los grupos usados en teoría de homología de los
usados en teoría de cohomología, convertimos los subíndices en superíndices.

\begin{definition}
Dados dos complejos de cocadenas $(C,d)$ y $(D,d')$, una \textbf{aplicación
de cocadenas} es un homomorfismo graduado $f\colon C \to D$ que conmuta con
el operador coborde (i.e. $d'\circ f=f\circ d$).
\end{definition}

Una aplicación de cocadenas $f$ siempre induce un homomorfismo en
cohomología,
\begin{diag}
f^*\colon H^*(C)\arrow[r] &H^*(D)\\[-8mm]
\left[c\right] \arrow[r,maps to]& \left[f(c)\right]
\end{diag}

\subsection{Cohomología singular}

\marginnote[-2.2cm]{
\begin{kaobox}[frametitle=Anillos]
Tanto la teoría de homología como cohomología se puede construir utilizando
anillos que no sean $\mb{Z}$; sin embargo, es más común encontrar casos
donde la cohomología se calcula para diferentes anillos (como nuestra
principal referencia, \cite{Vick94}).
\end{kaobox}
}

Sea $X$ un espacio topológico y $S_*(X)$ el grupo de cadenas singulares de
$X$. Dado un $n \in \mb{Z}$ y un anillo $R$, definimos el grupo de cocadenas
singulares $S^n(X;R)$ como el grupo formado por todos los homomorfismos de
la forma $\phi\colon S_n(X) \to R$.

\begin{theorem}
Sea $X$ un espacio topológico y $\p$ el operador borde asociado a $S_*(X)$.
El homomorfismo graduado $d=\{d^n\colon n \in \mb{Z}\}$,
\begin{diag}
d^n\colon S^n(X;R)\arrow[r] &S^{n+1}(X;R)\\[-8mm]
\phi \arrow[r,maps to]& \phi\circ \p_n
\end{diag}
define un operador coborde. Por tanto, $(S^*(X;R),d)$ es un complejo de
cocadenas.
\end{theorem}

Dada una aplicación continua $f\colon X \to Y$, podemos considerar la
aplicación inducida en cocadenas,
\begin{diag}
f^\#\colon S^n(X;R)\arrow[r] &S^n(Y;R)\\[-8mm]
\phi \arrow[r,maps to]& \phi\circ f
\end{diag}
que induce a su vez un homomorfismo en cohomología, $f^*\colon H^*(Y;R) \to
H^*(X;R)$. Si bien esta aplicación se asemeja al \emph{pull-back} de
geometría diferencial, es importante notar que $f^*$ actúa sobre clases y
no aplicaciones.

\section{Productos en cohomología}
En ciertos contextos, las teorías de cohomología se consideran más poderosas
que las teorías de homología. Esto se debe entre otras cosas a que podemos
dotar a $H^*(X)$ de una estructura de anillo, mediante la cual podemos
separar espacios topológicos que los grupos de homología no pueden
distinguir.

Sean $p,q \geq 0$ y $\phi\colon \sigma_{p+q} \to X$ un símplice singular.
Definimos
\begin{align*}
\phi^{(p)}(t_0,\dots,t_p)=\phi(t_0,\dots,t_p,0,\dots,0);\\
\phi^({q})(t_0,\dots,t_q)=\phi(0,\dots,0,t_0,\dots,t_q)
\end{align*}

\begin{definition}
Sean $c \in S^p(X;R)$ y $d \in S^q(X)$. Se define el producto de $c$ y $d$
como la $(p+q)$-cocadena $c\smile d \in S^{p+q}(X;R)$ dada por
\[(c\smile d)(\phi):=c(\phi^{(p)})d(\phi_{(q)})\]
La ley de composición interna $\smile$ recibe el nombre de \textbf{operador
copa}, e induce una estructura de anillo en $H^*(X;R)$.
\end{definition}

\subsection{Dualidad de Poincaré}
\begin{theorem}
Sea $M$ una variedad sin borde, compacta, conexa y orientable de dimensión
$n$. Si $s\colon M \to T$ es una orientación y $z$ es la
clase fundamental asociada a $M$, el homomorfismo
\begin{diag}
D\colon H^k(M;R)\arrow[r] &H_{n-k}(M;R)\\[-8mm]
\left[x\right] \arrow[r,maps to]& \left[x\frown z\right]
\end{diag}
es un isomorfismo para todo $k$.
\end{theorem}

\begin{example}[\cite{MSE}]
Sean $X=S^2\times S^2$ e $Y$ la suma conexa de dos copias de $\mb{CP}^2$.
Por un lado, tenemos que $H_*(X)\cong H_*(Y)$. Sin embargo,
\begin{align*}
H^*(X;\mb{Z})\cong \frac{\mb{Z}[x,y]}{(x^2,y^2)}; &&
H^*(Y;\mb{Z})\cong \frac{\mb{Z}[x,y]}{(x^3,y^3,xy)}
\end{align*}
que son anillos diferentes.
\end{example}
