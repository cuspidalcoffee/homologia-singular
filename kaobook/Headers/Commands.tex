%_____Topología________________________________________________________________
\DeclareMathOperator{\con}{con}		%Envoltura convexa
\DeclareMathOperator{\id}{id}		%Identidad
\DeclareMathOperator{\cte}{Cte}		%Aplicación constante
\newcommand{\p}{\partial}			%Operador borde


%_____Álgebra lineal___________________________________________________________
\DeclareMathOperator{\Gl}{Gl}				%Grupo lineal
\DeclareMathOperator{\End}{End}				%Grupo de endomorfismos
\DeclareMathOperator{\rk}{rk}				%Rango
\DeclareMathOperator{\im}{Im}				%Imagen
\newcommand{\m}[2]{\mathcal{M}_{#1}(#2)}	%Anillo de matrices
\newcommand{\la}{\left\langle}				%Antilambda izquierda
\newcommand{\ra}{\right\rangle}				%Antilambda derecha
\DeclareMathOperator{\sgn}{sgn}				%Signatura
\DeclareMathOperator{\ord}{ord}				%Orden

%____Tipografías_______________________________________________________________
\newcommand{\mb}[1]{\mathbb{#1}}		%Blackboard
\newcommand{\mc}[1]{\mathcal{#1}}		%Calligraphic
\newcommand{\ms}[1]{\mathscr{#1}}		%Script
\newcommand{\mf}[1]{\mathfrak{#1}}		%Fraktur
\newcommand{\bs}[1]{\boldsymbol{#1}}	%Bold
\newcommand{\eps}{\varepsilon}			%Épsilon

%_____Letra ш__________________________________________________________________
\DeclareFontFamily{U}{wncy}{}
\DeclareFontShape{U}{wncy}{m}{n}{<->wncyr10}{}
\DeclareSymbolFont{mcy}{U}{wncy}{m}{n}
\DeclareMathSymbol{\Sh}{\mathord}{mcy}{"58}

%_____Escribir encima de un símbolo_____________________________________________
\newcommand{\arriba}[2]{\stackrel{\mathclap{#1}}{#2}}
